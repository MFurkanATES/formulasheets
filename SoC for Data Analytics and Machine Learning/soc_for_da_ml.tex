

% ETH Systems-on-chip for Data Analytics and Machine Learning 2020
% ===========================================================================
% @Author: Noah Huetter
% @Date:   2020-02-18 17:26:28
% @Last Modified by:   noah
% @Last Modified time: 2020-02-18 18:40:54
% ---------------------------------------------------------------------------

\documentclass[a4paper, fontsize=8pt, landscape, DIV=1]{scrartcl}
\usepackage{lastpage}
\usepackage{hyperref}
% Include general settings and customized commands
%
% General packages and settings
% ===========================================================================
% Author:			Silvano Cortesi (cortesis@student.ethz.ch)
% Version:			1.2
% Last changed:		03.01.2018
%
% ---------------------------------------------------------------------------




\usepackage[german]{babel} %choose your language \usepackage[german]{babel}
%\usepackage[T1]{fontenc}
\usepackage[utf8]{inputenc}
\usepackage{fancyhdr}
%\usepackage{lastpage}
%\usepackage{lmodern}
\usepackage{enumerate}
%\usepackage{float} % for positioning of figures
\usepackage[landscape, margin=1cm]{geometry}
\usepackage[dvipsnames]{xcolor}
\usepackage{pdfpages}


%% Math %%
\usepackage{todonotes}
\usepackage{amscd}
\usepackage{blindtext}
\usepackage{enumitem}
\usepackage{multicol}
\usepackage{parskip}
\usepackage{empheq}
\usepackage{amsmath}
\usepackage{amsfonts}
\usepackage{amssymb}
\usepackage{amsthm}
%\usepackage{dsfont}
%\usepackage{esint} % provides \oiint
\usepackage{mathrsfs}
%\usepackage{trfsigns}
%\numberwithin{equation}{subsection}
%\usepackage{numprint}

%% Graphics & Charts %%
\usepackage{graphicx}
%\usepackage{pdfpages}
%\usepackage{booktabs}
\usepackage{array}
%\usepackage{paralist}
%\usepackage{framed}
%\usepackage{trfsigns}
\usepackage{tikz}
%\usepackage[lofdepth,lotdepth]{subfig}
%\usepackage{tikz}  %Graphen zeichnen
%\usetikzlibrary{decorations.pathmorphing}
%\usetikzlibrary{arrows.meta,arrows}
%\usepackage{pgfplots}
%% General Settings %%
%\setlength{\parindent}{0px}
%\setkomafont{captionlabel}{\normalfont\bfseries}
\usepackage{wrapfig}
\usepackage{color,soul}

%\pagestyle{fancy}
%\lfoot{\tiny \today}
%\rfoot{\thepage\  / \pageref{LastPage}}
%\cfoot{}
%\renewcommand{\footrulewidth}{0.4pt}

%% provides command \uline{} for underlining words
%\usepackage{ulem}

%% colour headings
%\usepackage{color}
%\definecolor{bluen}{cmyk}{1,0.5,0,0}
%\definecolor{bloodorange}{cmyk}{0,.92,1,.2}
%\addtokomafont{section}{\color{bloodorange}}
%\addtokomafont{subsection}{\color{bloodorange}}
%\addtokomafont{subsubsection}{\color{bloodorange}}
%\addtokomafont{paragraph}{\small\color{bloodorange}}
%\addtokomafont{subparagraph}{\small\color{bloodorange}}

%% Signs & Special Formating %%
%\usepackage{ulem} %normalem: \emph{Text} is italic again.
%\usepackage{multicol,multirow}
%\usepackage{tabularx}
%\usepackage{stackrel}
%\usepackage{makeidx}
%\usepackage{mparhack} % bessere margiale bei seitenumbruch

% make document compact
\usepackage[compact]{titlesec}
\titlespacing{\section}{0pt}{*0}{*0}
\titlespacing{\subsection}{0pt}{*0}{*0}
\titlespacing{\subsubsection}{0pt}{*0}{*0}

\parindent 0pt
\pagestyle{empty}
\setlength{\unitlength}{1cm}
\setlist{leftmargin = *}

%include also newer PDF
% \pdfminorversion=6

% Set the color of your style
% Avaiable are: Apricot, Aquamarine, Bittersweet, Black, Blue, blue, BlueGreen, BlueViolet, BrickRed, Brown, BurntOrange, CadetBlue, CarnationPink, Cerulean, CornflowerBlue, Cyan, Dandelion, DarkOrchid, Emerald, ForestGreen, Fuchsia, Goldenrod, Gray, Green, GreenYellow, JungleGreen, Lavender, ... (more at: http://en.wikibooks.org/wiki/LaTeX/Colors)
\def\StyleColor{MidnightBlue}

%
% General commands
% ===========================================================================
% Author:			Silvano Cortesi (cortesis@student.ethz.ch)
% Version:			1.2
% Last changed:		03.01.2018
%
% ---------------------------------------------------------------------------

%..ROEMISCHE_ZAHLEN
	\newcommand{\Roe}[1]{\uppercase\expandafter{\romannumeral #1 }}

%..ZAHLENMENGEN
	\newcommand{\N}{\mathbb{N}}
	\newcommand{\Z}{\mathbb{Z}}
	\newcommand{\Q}{\mathbb{Q}}
	\newcommand{\R}{\mathbb{R}}
	\newcommand{\real}{\R}
	\newcommand{\C}{\mathbb{C}}
	\newcommand{\complex}{\C}
	\newcommand{\0}{\mathbb{O}}
	\newcommand{\F}{\mathbb{F}}
	\newcommand{\K}{\mathbb{K}}
    \newcommand{\angstrom}{\textup{\AA}}
    
%..PFEILE
	\renewcommand{\leadsto}{\Longrightarrow}
	\newcommand{\leftrightleadsto}{\Longleftrightarrow}

%..SCHRIFT
	\newcommand{\mbf}[1] {\mathbf{#1}}
	\newcommand{\mrm}[1] {\mathrm{#1}}
  \renewcommand{\phi}{\varphi}

%..VEKTOREN
	\newcommand{\Ul} {\underline}
	\newcommand{\vEx} {\vec{e}_x}
	\newcommand{\vEy} {\vec{e}_y}
	\newcommand{\vEz} {\vec{e}_z}
	\newcommand{\vEq} {\vec{e_1}}
	\newcommand{\vEw} {\vec{e_2}}
	\newcommand{\vEe} {\vec{e_3}}
	\newcommand{\transpose} {^{\text{T}}}
	\newcommand{\vect}[1]{\boldsymbol{#1}}
	
%..MATRIX
    \newcommand{\MATR}[1]{ \displaystyle \left[ \begin{matrix} #1 \end{matrix} \right]}
    \newcommand{\MATRABS}[1]{ \displaystyle \left| \begin{matrix} #1 \end{matrix} \right|}


%..KOMPLEXE ZAHLEN
	\renewcommand{\Re}{\text{Re}\,}
	\renewcommand{\Im}{\text{Im}\,}

%..OPERATOREN
	\DeclareMathOperator{\grad}{grad}
	\renewcommand{\div}{\text{div}\,}
    	\DeclareMathOperator{\rot}{rot}
    	\DeclareMathOperator{\divg}{div}
    	\DeclareMathOperator{\Tr}{Tr}
    	\DeclareMathOperator{\const}{const}
	\DeclareMathOperator{\imag}{i}
	\newcommand{\Lapl}{\hbox{\footnotesize{$\Delta$}}}

%..DIFFERENTIALRECHNUNG
	\newcommand{\Dx} {\,\mathrm{d}}
	\newcommand{\abl}[1] {\frac{\mathrm{d}}{\mathrm{d}#1}}
	\newcommand{\Abl}[2] {\frac{\mathrm{d}#1}{\mathrm{d}#2}}
	\newcommand{\ablq}[1] {\frac{\mathrm{d^2}}{\mathrm{d}#1^2}}
	\newcommand{\Ablq}[2] {\frac{\mathrm{d^2}#1}{\mathrm{d}#2^2}}
	\newcommand{\pabl}[1] {\frac{\partial}{\partial#1}}
	\newcommand{\pablq}[1] {\frac{\partial^2}{\partial#1^2}}
	\newcommand{\Pabl}[2] {\frac{\partial#1}{\partial#2}}
	\newcommand{\Pablq}[2] {\frac{\partial^2#1}{\partial#2^2}}

%..INTEGRALRECHNUNG
	\newcommand{\dint}{\displaystyle{\int}}
	\newcommand{\intab}{\int^b_a}
	\newcommand{\intinf}{\int_{-\infty}^\infty}
  \newcommand{\Int}{\int\displaylimits}
	\newcommand{\dintab}{\displaystyle{\int^b_a}}
	\newcommand{\dintpi}{\displaystyle{\int^{\pi}_{-\pi}}}
	\newcommand{\dintzpi}{\displaystyle{\int^{2\pi}_{\mbox{-}2\pi}}}
	\newcommand{\dA}{\hspace{4pt}\mathrm{d}A}
	\newcommand{\dx}{\hspace{4pt}\mathrm{d}x}
	\newcommand{\dy}{\hspace{4pt}\mathrm{d}y}
	\newcommand{\dz}{\hspace{4pt}\mathrm{d}z}
	\newcommand{\dr}{\hspace{4pt}\mathrm{d}r}
	\newcommand{\ds}{\hspace{4pt}\mathrm{d}s}
	\newcommand{\dS}{\hspace{4pt}\mathrm{d}S}
	\newcommand{\dt}{\hspace{4pt}\mathrm{d}t}
	\newcommand{\dm}{\hspace{4pt}\mathrm{d}m}
	\newcommand{\dk}{\hspace{4pt}\mathrm{d}k}
	\newcommand{\dl}{\hspace{4pt}\mathrm{d}l}
	\newcommand{\du}{\hspace{4pt}\mathrm{d}u}
	\newcommand{\dv}{\hspace{4pt}\mathrm{d}v}
	\newcommand{\dV}{\hspace{4pt}\mathrm{d}V}
	\newcommand{\dphi}{\hspace{4pt}\mathrm{d}\varphi}
	\newcommand{\domega}{\hspace{4pt}\mathrm{d}\omega}
	\newcommand{\dvarsigma}{\hspace{4pt}\mathrm{d}\varsigma}
	\newcommand{\dtau}{\hspace{4pt}\mathrm{d}\tau}
	\newcommand{\dtheta}{\hspace{4pt}\mathrm{d}\vartheta}
	\newcommand{\dmu}{\hspace{4pt}\mathrm{d}\mu}
	\newcommand{\dxi}{\hspace{4pt}\mathrm{d}\xi}
	\newcommand{\deta}{\hspace{4pt}\mathrm{d}\eta}
	\newcommand{\dvecl}{\hspace{4pt}\mathrm{d}\vec{l}}
	\newcommand{\dvecS}{\hspace{4pt}\mathrm{d}\vec{S}}

%..LIMES
    \DeclareMathOperator{\limni}{\lim\limits_{n\to\infty}}
    \DeclareMathOperator{\limxi}{\lim\limits_{x\to\infty}}
    \DeclareMathOperator{\limho}{\lim\limits_{h\to0}}
    \newcommand{\limxai}[1]{\ensuremath{\lim\limits_{x\to #1}}}

%..SUMMEN
    \DeclareMathOperator{\sumni}{\sum_{n=0}^{\infty}}
    \newcommand{\sumnia}[1]{\ensuremath{\sum_{n=#1}^{\infty}}}


%..PARTIELLE ABLEITUNG
    \DeclareMathOperator{\partf}{\dfrac{\partial f}{\partial x}}
    \newcommand{\partfo}[1]{\ensuremath{\dfrac{\partial f}{\partial #1}}}
    \newcommand{\parto}[1]{\ensuremath{\dfrac{\partial }{\partial #1}}}
    \newcommand{\partt}[2]{\ensuremath{\dfrac{\partial^2 }{\partial #1\partial #2}}}
    \newcommand{\partq}[1]{\ensuremath{\dfrac{\partial^2 }{\partial #1^2}}}


%..ENUMERATION
    \newenvironment{abc}{\begin{enumerate}[(a)]}{\end{enumerate}}
    \newenvironment{cabc}{\begin{compactenum}[(a)]}{\end{compactenum}}
    \newenvironment{romanenum}{\begin{enumerate}[i.]}{\end{enumerate}}
    \newenvironment{cromanenum}{\begin{compactenum}[i.]}{\end{compactenum}}

%..FUNCTIONS
    \DeclareMathOperator{\arsinh}{arsinh}
    \DeclareMathOperator{\arcosh}{arcosh}
    \DeclareMathOperator{\artanh}{artanh}
    \DeclareMathOperator{\arcoth}{arcoth}
    \DeclareMathOperator{\arccot}{arccot}
    \DeclareMathOperator{\Arg}{Arg}
    \DeclareMathOperator{\Log}{Log}
    \newcommand{\dis}[1]{\hspace{#1cm}}
    \newcommand{\abs}[1]{\ensuremath{\left\vert#1\right\vert}}
    \newcommand{\attention}{\raisebox{-1pt}{{\makebox[1.6em][c]{\makebox[0pt][c]{\raisebox{.13em}{\small!}}\makebox[0pt][c]{\color{red}\Large$\bigtriangleup$}}}}}
    \DeclareMathOperator{\meq}{\stackrel{!}{=}}
    
%..GRAPHICS
	\newcommand{\cgraphic}[2]{\begin{center}\includegraphics[width=#1\columnwidth,keepaspectratio]{#2}\end{center}}
	\newcommand{\mgraphic}[2]{\begin{wrapfigure}{l}{#1\linewidth}\includegraphics[width=\linewidth]{#2}\end{wrapfigure}}
    
% section color box
\setkomafont{section}{\mysection}
\newcommand{\mysection}[1]{%
    \Large\sffamily\bfseries%
    \setlength{\fboxsep}{0cm}%already boxed
    \colorbox{\StyleColor!40}{%
        \begin{minipage}{\linewidth}%
            \vspace*{2pt}%Space before
            #1
            \vspace*{-1pt}%Space after
        \end{minipage}%
    }}

%subsection color box
\setkomafont{subsection}{\mysubsection}
\newcommand{\mysubsection}[1]{%
    \normalsize \sffamily\bfseries%
    \setlength{\fboxsep}{0cm}%already boxed
    \colorbox{\StyleColor!20}{%
        \begin{minipage}{\linewidth}%
            \vspace*{2pt}%Space before
             #1
            \vspace*{-1pt}%Space after
        \end{minipage}%
    }}

% highlighter
\newcommand{\hilight}[1]{\colorbox{\StyleColor}{#1}}
\newcommand{\highlighty}[1]{%
  \setlength{\fboxsep}{0pt}\colorbox{yellow!100}{\ensuremath{#1}}}

\newcommand{\highlightg}[1]{%
  \setlength{\fboxsep}{0pt}\colorbox{green!100}{\ensuremath{#1}}}

\newcommand{\highlightbg}[1]{%
   \colorbox{green!100}{$\displaystyle #1$}}  

% equation box        
\newcommand{\eqbox}[1]{\setlength{\fboxrule}{1mm}\fcolorbox{\StyleColor}{white}{\hspace{0.5em}$\displaystyle#1$\hspace{0.5em}}}

%center equationbox
\newcommand{\ceqbox}[1]{\vspace*{4pt} \begin{center}\eqbox{#1}\end{center}\vspace*{4pt}}


% This package makes formulas a bit more compact but less beautiful
% \usepackage{newtxtext,newtxmath}

% scala language description
\lstdefinelanguage{BNF}{%
    alsoletter={-},%
    sensitive,%
}[keywords,comments]%

\lstset{%
    basicstyle=\ttfamily,%
%    language=P4,%
    aboveskip=3mm,%
    belowskip=3mm,%
    fontadjust=true,%
%    columns=[c]fixed,%
    keepspaces=true,%
%    commentstyle=\itshape,%
    frame=single,
    keywordstyle=\bfseries,%
    captionpos=b,%
    framerule=0.3pt,%
    firstnumber=0,%
    numbersep=1.5mm,%
    numberstyle=\tiny,%
    breaklines=true,
    postbreak=\mbox{\textcolor{red}{$\hookrightarrow$}\space},
}

% \bibliography{semiconductordevices}
% \bibliographystyle{ieeetr}
\medmuskip=1mu

%change page style for header
\pagestyle{fancy}
\footskip 20pt

% Uncomment this line to make formulasheet ultra compact
% This removes
% - list of variables
% \newcommand{\makeultracompact}{irrelevant}
\let\makeultracompact\undefined

% Make stuff ultra compact if so desired
\ifdefined\makeultracompact
  \setlength{\parskip}{0pt}
  \setlength{\abovedisplayskip}{0pt}
  \setlength{\belowdisplayskip}{0pt}
  \setlength{\abovedisplayshortskip}{0pt}
  \setlength{\belowdisplayshortskip}{0pt}
\else
\fi
 
% -----------------------------------------------------------------------
\IfFileExists{../build/revision.tex}{
  \input{../build/revision.tex}
  \rhead{Compiled: \compiledate \hspace{1em} on: \hostname \hspace{1em} git-sha: \revision \hspace{1em} Noah Huetter}
}{\rhead{Noah Huetter}}

\ifdefined\makeultracompact
  \lhead{ETH Systems-on-chip for Data Analytics and Machine Learning 2020 \hspace{1em}compact version}
\else
  \lhead{ETH Systems-on-chip for Data Analytics and Machine Learning 2020}
\fi
\chead{\thepage}
\cfoot{}
\headheight 17pt \headsep 10pt
\title{ETH Systems-on-chip for Data Analytics and Machine Learning 2020}
\author{Noah Huetter}

\date{\today}
\begin{document}

\setcounter{page}{0}
\setcounter{secnumdepth}{2} %no enumeration of sections
\begin{multicols*}{4}
	\section*{Disclaimer}
	This summary is part of the lecture ``Systems-on-chip for Data Analytics and Machine Learning'' (227-0150-00L) by Prof. Dr. Luca Benini (FS20). \\[6pt]
	Please report errors to \href{mailto:huettern@student.ethz.ch}{huettern@student.ethz.ch} such that others can benefit as well.\\[6pt]	
  The upstream repository can be found at \href{https://github.com/noah95/formulasheets}{https://github.com/noah95/formulasheets}
	\vfill\null
  \columnbreak
  %%%%%%%%%%%%%%%%%%%%%%%%%%%%%
  \setcounter{tocdepth}{2}
  \tableofcontents
  \vfill\null
  %\columnbreak
  %%%%%%%%%%%%%%%%%%%%%%%%%%%%%
	\pagebreak
  \maketitle 
  \setcounter{page}{1}
  \thispagestyle{fancy}

  % ---------------------------------------------------------------------------
  \section{Architecture review, MCUs}
  % ---------------------------------------------------------------------------
  \subsection{Datacenters}
  \subsubsection{Cloud Service Models}
  \begin{outline}
    \1 Software as a Service (Saas)
      \2 Provider licenses applications to users as a service
      \2 Avoid costs of installation, maintenance, patches, ...
    \1 Platform as a Service (Paas)
      \2 Provider offers software platform for building applications
      \2 e.g. Google's App-Engine
      \2 Avoid worrying about scalability of platform
    \1 Infrastructure as a Service (Iaas)
      \2 Povider offers raw computing, storage and network
      \2 e.g. Amazon AWS
      \2 Avoid buying servers and estimating resource needs
  \end{outline}

  \subsubsection{Building blocks of moden data centers}
  \begin{outline}
    \1 Network Switch
    \1 Rack
    \1 Server Racks
    \1 Cluster Switch
  \end{outline}
  Rack of servers (so called commodity servers) consist of a modular design with
  top-of-rack switch, power, network and storage. The ToR is the link aggregate to 
  the next level.

  A data center is not just a collectino of servers, it's a "small internet". It is
  administered as a single domain that does not have to be compatible with the "outside
  world". No need for international standards.

  There exists specialized machine learning cloud hardware.

  \subsubsection{Performance Metrics}
  \begin{outline}
    \1 Throughput
      \2 Requests per second
      \2 Concurrent users
      \2 Gbytes/sec processed
    \1 Latency
      \2 Execution time
      \2 Per request latency
  \end{outline}

  \subsubsection{Tail Latency}
  Output depends on all servers finishing. Probability of a slow server $P_\text{slow}$.
  Job on $N$ servers finishes when the last server is done. Even if $P_\text{slow}$ is
  very small, $P_\text{jobFsat}$ is small if $N$ is large.
  \begin{empheq}[box=\eqbox]{equation*}
    \begin{gathered}
      P_\text{jobFast} = (1-P_\text{slow})^N
    \end{gathered}
  \end{empheq}
  \cgraphic{1.0}{img/tail.jpg}

  \subsubsection{TCO: Total Cost of Ownership}
  TCO = capital (CapEx) + operational (OpEx) expenses. CapEx: building, generators, HW.
  OpEx: Elec., repairs, insurance. Users perspective: CapEx: cost of long term leases on HW and
  services. OpEx: Pay per use cost on HW and services

  \subsubsection{Reliability}
  \begin{outline}
    \1 Failure in time (FIT)
      \2 Failures per billion hours of operation
    \1 Mean time to failure (MTTF)
      \2 Time to produce first incorrect output
    \1 Mean time to repair (MTTR)
      \2 Time to detect and repair a failure
  \end{outline}

  Steady state availability = MTTF/(MTTF + MTTR).

  \subsubsection{Multicore}
   \begin{tabularx}{\linewidth}{l c c c}
    \hline
    {} & Single & Dual & Quad \\ \hline
    Core area & $A$ & $\approx A/2$ & $\approx A/4$ \\
    Core power & $W$ & $\approx W/2$ & $\approx W/4$ \\
    Chip power & $W+O$ & $W+O'$ & $W+O''$ \\
    Core performance & $P$ & $0.9P$ & $0.8P$ \\
    Chip performance & $P$ & $1.8P$ & $3.2P$ \\ \hline
  \end{tabularx}

  \subsubsection{Amdahl's Law}
  Not all operations can run in parallel, hence speedup is limited. $f$ fraction 
  that can run in parallel.
  \begin{empheq}[box=\eqbox]{equation*}
    \begin{align}
      \text{Speedup} &= \frac{1}{(1-f)+\frac{f}{n}} & \lim_{n\to\infty}&\frac{1}{1-f+\frac{f}{n}}=\frac{1}{1-f}
    \end{align}
  \end{empheq}
  Amdahl's Law ignores power cost of $n$ cores. More cores resupts in each core being slower. 
  Parallel speedup $< n$. Seiral portion takes longer, also, inerconnect and scaling overhead.
  Fixed power budget forces slow cores, serial code quickly dominates.
  \cgraphic{1.0}{img/amdahl.png}

  \subsection{Computing Systems Performance}
  \subsubsection{Instruction Count and CPI}
  CPI: Cycles per instruction. CPU time = instruction count x CPI x clock cycle time.
  CPI is determined by program, ISA and compiler. Different instructions have diferent CPI.

  \subsubsection{Performance}
  IC: Instruction count.
  Depends on
  \begin{outline}
    \1 Algorithm: affects IC, possibly CPI
    \1 Programming language, affects IC, CPI
    \1 Compiler: affects IC, CPI
    \1 Instruction set architecture: affects IC, CPI, Tc
  \end{outline}

  \subsubsection{Power}
  \begin{empheq}[box=\eqbox]{equation*}
    \begin{gathered}
      \text{Power} = \text{Capactivice load} \times \text{Voltage}^2 \times \text{Frequency}
    \end{gathered}
  \end{empheq}

  \subsubsection{Multiprocessors}
  More than one processor per chip. Requires explicitly parallel programming.
  Hard to rogramm for performance, balance load and optimize synchronization.

  \subsection{Processor Architecture}
  CPU performance factors are determined by instruction count, CPI and cycle time.

  \subsubsection{Instruction Execution}
  Programm counter (PC) points to instruction memory to fetch instructions. Register
  numbers point to the register file to read registers. Depending on instruction class,
  use the ALU for arithmetic results, memory address calculation or branch compare. 
  Access data memory for load/store and increment PC by target address of 4 (4 byte instruction
  words).
  \cgraphic{1.0}{img/archoverview.png}

  \subsubsection{R-format instructions}
  Read two register from register file, perform arithmetic/logical operation and
  write back to register.

  \subsubsection{Load/Store Instructions}
  Read register operands, calculate address using offset, load memory to register
  or vice versa.

  \subsubsection{Branch Instructions}
  Read register, compare operands, calculate target address.

  \subsubsection{Pipelining}
  RISC-V is designed for pipelining: All instructions 32-bit, few regular instruction
  format, load/store addressing.

  \subsubsection{Hazards}
  Situations that precent starting the next instruction in the next cycle.
  \begin{outline}
    \1 Structure hazard
      \2 A required resource is busy
      \2 Fix: requires separate instruction/data memories
    \1 Data hazard
      \2 Need to wait for precious instruction to complete its data read/write
      \2 Fix: Forward data to next op without storing in register
      \2 Fix: Code scheduling to avoid stalls
    \1 Control hazard
      \2 Deciding on control action depends on precious instruction
      \2 Fix: Branch prediction
  \end{outline}

  \subsubsection{Branch Prediction}
  Predict outcome of branch and only stall if prediction is wrong. RISC-V can predict branches
  not taken. Static prediction based on typical branch behaviour. Dynamic measures actual branch
  behaviour, e.g. record recent history of each branch. Assume future behav. will continue the trend,
  when wrong, stall while re-fetching. 2-Bit predictor changes prediction only on two successive
  mispredictions.
  \cgraphic{1.0}{img/pipeline.png}

  \subsection{Exploiting Memory Hieararchy}
  Programs access a small portion of their address space at any time. Temporal locality aims at items
  that are accessed recently to be likely accessed again soon (e.g. loop). Spatial locality towards
  items near recently accessed (e.g. sequential instructions, array data).

  \subsubsection{Locality}
  Copy recently accessed (and nearby) items from disk to smaller DRAM (main memory) and from there 
  to smaller SRAM (cache attahched to CPU). If accessed data is present, \textbf{hit}, else \textbf{miss}.
  Hit ratio: hits/accesses.

  \subsubsection{Cache Memory}
  \begin{outline}
    \1 Direct Mapped Cache
      \2 Location determined by address
      \2 Direct mapped: only one choice
      \2 Store tags (high order bits of source data) and valid flag if there is data at that location
    \1 Larger block size
      \2 Store tag, index and offset
      \2 Larger blocks should reduce miss rate but larger miss penalty
    \1 Associative Cache
      \2 Allow a given block to go in any cache entry
      \2 Requires all entries to be searched at once
    \1 $n$-way set associative
      \2 Each set contains $n$ entries
      \2 Blcok number determines which set
      \2 Search all entries in a given set at once
  \end{outline}

  \subsubsection{AMAT: Average memory access time}
  AMAT = Hit time + miss rate x miss penalty

  \subsubsection{Replacement Policy}
  Direct mapped: no choice. Set associative: Prefer non-valid entry, if there is one. Otherwise,
  choose among entries in the set. Least recently used (LRU): Choose the one unused for the longest time.
  Random: Gives approx. same performance as LRU.

  % ---------------------------------------------------------------------------
  \section{ML Concepts, ML Hardware}
  % ---------------------------------------------------------------------------


  % ---------------------------------------------------------------------------
  \section{Out of Order Processors}
  % ---------------------------------------------------------------------------


  % ---------------------------------------------------------------------------
  \section{Multicore Processors}
  % ---------------------------------------------------------------------------


  % ---------------------------------------------------------------------------
  \section{Vector Processors, Multithreading, Virtualizing}
  % ---------------------------------------------------------------------------


  % ---------------------------------------------------------------------------
  \section{GP-GPUs}
  % ---------------------------------------------------------------------------


  % ---------------------------------------------------------------------------
  \section{Heterogeneous SoCs}
  % ---------------------------------------------------------------------------



    
\end{multicols*}

\setcounter{secnumdepth}{2}
\end{document}
