

% Summary semiconductor devices D-ITET
% ===========================================================================
% @Author: Noah Huetter
% @Date:   2019-02-20 17:26:28
% @Last Modified by:   noah
% @Last Modified time: 2019-03-13 15:26:54
% ---------------------------------------------------------------------------

\documentclass[a4paper, fontsize=8pt, landscape, DIV=1]{scrartcl}
\usepackage{lastpage}
\usepackage{hyperref}
% Include general settings and customized commands
%
% General packages and settings
% ===========================================================================
% Author:			Silvano Cortesi (cortesis@student.ethz.ch)
% Version:			1.2
% Last changed:		03.01.2018
%
% ---------------------------------------------------------------------------




\usepackage[german]{babel} %choose your language \usepackage[german]{babel}
%\usepackage[T1]{fontenc}
\usepackage[utf8]{inputenc}
\usepackage{fancyhdr}
%\usepackage{lastpage}
%\usepackage{lmodern}
\usepackage{enumerate}
%\usepackage{float} % for positioning of figures
\usepackage[landscape, margin=1cm]{geometry}
\usepackage[dvipsnames]{xcolor}
\usepackage{pdfpages}


%% Math %%
\usepackage{todonotes}
\usepackage{amscd}
\usepackage{blindtext}
\usepackage{enumitem}
\usepackage{multicol}
\usepackage{parskip}
\usepackage{empheq}
\usepackage{amsmath}
\usepackage{amsfonts}
\usepackage{amssymb}
\usepackage{amsthm}
%\usepackage{dsfont}
%\usepackage{esint} % provides \oiint
\usepackage{mathrsfs}
%\usepackage{trfsigns}
%\numberwithin{equation}{subsection}
%\usepackage{numprint}

%% Graphics & Charts %%
\usepackage{graphicx}
%\usepackage{pdfpages}
%\usepackage{booktabs}
\usepackage{array}
%\usepackage{paralist}
%\usepackage{framed}
%\usepackage{trfsigns}
\usepackage{tikz}
%\usepackage[lofdepth,lotdepth]{subfig}
%\usepackage{tikz}  %Graphen zeichnen
%\usetikzlibrary{decorations.pathmorphing}
%\usetikzlibrary{arrows.meta,arrows}
%\usepackage{pgfplots}
%% General Settings %%
%\setlength{\parindent}{0px}
%\setkomafont{captionlabel}{\normalfont\bfseries}
\usepackage{wrapfig}
\usepackage{color,soul}

%\pagestyle{fancy}
%\lfoot{\tiny \today}
%\rfoot{\thepage\  / \pageref{LastPage}}
%\cfoot{}
%\renewcommand{\footrulewidth}{0.4pt}

%% provides command \uline{} for underlining words
%\usepackage{ulem}

%% colour headings
%\usepackage{color}
%\definecolor{bluen}{cmyk}{1,0.5,0,0}
%\definecolor{bloodorange}{cmyk}{0,.92,1,.2}
%\addtokomafont{section}{\color{bloodorange}}
%\addtokomafont{subsection}{\color{bloodorange}}
%\addtokomafont{subsubsection}{\color{bloodorange}}
%\addtokomafont{paragraph}{\small\color{bloodorange}}
%\addtokomafont{subparagraph}{\small\color{bloodorange}}

%% Signs & Special Formating %%
%\usepackage{ulem} %normalem: \emph{Text} is italic again.
%\usepackage{multicol,multirow}
%\usepackage{tabularx}
%\usepackage{stackrel}
%\usepackage{makeidx}
%\usepackage{mparhack} % bessere margiale bei seitenumbruch

% make document compact
\usepackage[compact]{titlesec}
\titlespacing{\section}{0pt}{*0}{*0}
\titlespacing{\subsection}{0pt}{*0}{*0}
\titlespacing{\subsubsection}{0pt}{*0}{*0}

\parindent 0pt
\pagestyle{empty}
\setlength{\unitlength}{1cm}
\setlist{leftmargin = *}

%include also newer PDF
% \pdfminorversion=6

% Set the color of your style
% Avaiable are: Apricot, Aquamarine, Bittersweet, Black, Blue, blue, BlueGreen, BlueViolet, BrickRed, Brown, BurntOrange, CadetBlue, CarnationPink, Cerulean, CornflowerBlue, Cyan, Dandelion, DarkOrchid, Emerald, ForestGreen, Fuchsia, Goldenrod, Gray, Green, GreenYellow, JungleGreen, Lavender, ... (more at: http://en.wikibooks.org/wiki/LaTeX/Colors)
\def\StyleColor{MidnightBlue}

%
% General commands
% ===========================================================================
% Author:			Silvano Cortesi (cortesis@student.ethz.ch)
% Version:			1.2
% Last changed:		03.01.2018
%
% ---------------------------------------------------------------------------

%..ROEMISCHE_ZAHLEN
	\newcommand{\Roe}[1]{\uppercase\expandafter{\romannumeral #1 }}

%..ZAHLENMENGEN
	\newcommand{\N}{\mathbb{N}}
	\newcommand{\Z}{\mathbb{Z}}
	\newcommand{\Q}{\mathbb{Q}}
	\newcommand{\R}{\mathbb{R}}
	\newcommand{\real}{\R}
	\newcommand{\C}{\mathbb{C}}
	\newcommand{\complex}{\C}
	\newcommand{\0}{\mathbb{O}}
	\newcommand{\F}{\mathbb{F}}
	\newcommand{\K}{\mathbb{K}}
    \newcommand{\angstrom}{\textup{\AA}}
    
%..PFEILE
	\renewcommand{\leadsto}{\Longrightarrow}
	\newcommand{\leftrightleadsto}{\Longleftrightarrow}

%..SCHRIFT
	\newcommand{\mbf}[1] {\mathbf{#1}}
	\newcommand{\mrm}[1] {\mathrm{#1}}
  \renewcommand{\phi}{\varphi}

%..VEKTOREN
	\newcommand{\Ul} {\underline}
	\newcommand{\vEx} {\vec{e}_x}
	\newcommand{\vEy} {\vec{e}_y}
	\newcommand{\vEz} {\vec{e}_z}
	\newcommand{\vEq} {\vec{e_1}}
	\newcommand{\vEw} {\vec{e_2}}
	\newcommand{\vEe} {\vec{e_3}}
	\newcommand{\transpose} {^{\text{T}}}
	\newcommand{\vect}[1]{\boldsymbol{#1}}
	
%..MATRIX
    \newcommand{\MATR}[1]{ \displaystyle \left[ \begin{matrix} #1 \end{matrix} \right]}
    \newcommand{\MATRABS}[1]{ \displaystyle \left| \begin{matrix} #1 \end{matrix} \right|}


%..KOMPLEXE ZAHLEN
	\renewcommand{\Re}{\text{Re}\,}
	\renewcommand{\Im}{\text{Im}\,}

%..OPERATOREN
	\DeclareMathOperator{\grad}{grad}
	\renewcommand{\div}{\text{div}\,}
    	\DeclareMathOperator{\rot}{rot}
    	\DeclareMathOperator{\divg}{div}
    	\DeclareMathOperator{\Tr}{Tr}
    	\DeclareMathOperator{\const}{const}
	\DeclareMathOperator{\imag}{i}
	\newcommand{\Lapl}{\hbox{\footnotesize{$\Delta$}}}

%..DIFFERENTIALRECHNUNG
	\newcommand{\Dx} {\,\mathrm{d}}
	\newcommand{\abl}[1] {\frac{\mathrm{d}}{\mathrm{d}#1}}
	\newcommand{\Abl}[2] {\frac{\mathrm{d}#1}{\mathrm{d}#2}}
	\newcommand{\ablq}[1] {\frac{\mathrm{d^2}}{\mathrm{d}#1^2}}
	\newcommand{\Ablq}[2] {\frac{\mathrm{d^2}#1}{\mathrm{d}#2^2}}
	\newcommand{\pabl}[1] {\frac{\partial}{\partial#1}}
	\newcommand{\pablq}[1] {\frac{\partial^2}{\partial#1^2}}
	\newcommand{\Pabl}[2] {\frac{\partial#1}{\partial#2}}
	\newcommand{\Pablq}[2] {\frac{\partial^2#1}{\partial#2^2}}

%..INTEGRALRECHNUNG
	\newcommand{\dint}{\displaystyle{\int}}
	\newcommand{\intab}{\int^b_a}
	\newcommand{\intinf}{\int_{-\infty}^\infty}
  \newcommand{\Int}{\int\displaylimits}
	\newcommand{\dintab}{\displaystyle{\int^b_a}}
	\newcommand{\dintpi}{\displaystyle{\int^{\pi}_{-\pi}}}
	\newcommand{\dintzpi}{\displaystyle{\int^{2\pi}_{\mbox{-}2\pi}}}
	\newcommand{\dA}{\hspace{4pt}\mathrm{d}A}
	\newcommand{\dx}{\hspace{4pt}\mathrm{d}x}
	\newcommand{\dy}{\hspace{4pt}\mathrm{d}y}
	\newcommand{\dz}{\hspace{4pt}\mathrm{d}z}
	\newcommand{\dr}{\hspace{4pt}\mathrm{d}r}
	\newcommand{\ds}{\hspace{4pt}\mathrm{d}s}
	\newcommand{\dS}{\hspace{4pt}\mathrm{d}S}
	\newcommand{\dt}{\hspace{4pt}\mathrm{d}t}
	\newcommand{\dm}{\hspace{4pt}\mathrm{d}m}
	\newcommand{\dk}{\hspace{4pt}\mathrm{d}k}
	\newcommand{\dl}{\hspace{4pt}\mathrm{d}l}
	\newcommand{\du}{\hspace{4pt}\mathrm{d}u}
	\newcommand{\dv}{\hspace{4pt}\mathrm{d}v}
	\newcommand{\dV}{\hspace{4pt}\mathrm{d}V}
	\newcommand{\dphi}{\hspace{4pt}\mathrm{d}\varphi}
	\newcommand{\domega}{\hspace{4pt}\mathrm{d}\omega}
	\newcommand{\dvarsigma}{\hspace{4pt}\mathrm{d}\varsigma}
	\newcommand{\dtau}{\hspace{4pt}\mathrm{d}\tau}
	\newcommand{\dtheta}{\hspace{4pt}\mathrm{d}\vartheta}
	\newcommand{\dmu}{\hspace{4pt}\mathrm{d}\mu}
	\newcommand{\dxi}{\hspace{4pt}\mathrm{d}\xi}
	\newcommand{\deta}{\hspace{4pt}\mathrm{d}\eta}
	\newcommand{\dvecl}{\hspace{4pt}\mathrm{d}\vec{l}}
	\newcommand{\dvecS}{\hspace{4pt}\mathrm{d}\vec{S}}

%..LIMES
    \DeclareMathOperator{\limni}{\lim\limits_{n\to\infty}}
    \DeclareMathOperator{\limxi}{\lim\limits_{x\to\infty}}
    \DeclareMathOperator{\limho}{\lim\limits_{h\to0}}
    \newcommand{\limxai}[1]{\ensuremath{\lim\limits_{x\to #1}}}

%..SUMMEN
    \DeclareMathOperator{\sumni}{\sum_{n=0}^{\infty}}
    \newcommand{\sumnia}[1]{\ensuremath{\sum_{n=#1}^{\infty}}}


%..PARTIELLE ABLEITUNG
    \DeclareMathOperator{\partf}{\dfrac{\partial f}{\partial x}}
    \newcommand{\partfo}[1]{\ensuremath{\dfrac{\partial f}{\partial #1}}}
    \newcommand{\parto}[1]{\ensuremath{\dfrac{\partial }{\partial #1}}}
    \newcommand{\partt}[2]{\ensuremath{\dfrac{\partial^2 }{\partial #1\partial #2}}}
    \newcommand{\partq}[1]{\ensuremath{\dfrac{\partial^2 }{\partial #1^2}}}


%..ENUMERATION
    \newenvironment{abc}{\begin{enumerate}[(a)]}{\end{enumerate}}
    \newenvironment{cabc}{\begin{compactenum}[(a)]}{\end{compactenum}}
    \newenvironment{romanenum}{\begin{enumerate}[i.]}{\end{enumerate}}
    \newenvironment{cromanenum}{\begin{compactenum}[i.]}{\end{compactenum}}

%..FUNCTIONS
    \DeclareMathOperator{\arsinh}{arsinh}
    \DeclareMathOperator{\arcosh}{arcosh}
    \DeclareMathOperator{\artanh}{artanh}
    \DeclareMathOperator{\arcoth}{arcoth}
    \DeclareMathOperator{\arccot}{arccot}
    \DeclareMathOperator{\Arg}{Arg}
    \DeclareMathOperator{\Log}{Log}
    \newcommand{\dis}[1]{\hspace{#1cm}}
    \newcommand{\abs}[1]{\ensuremath{\left\vert#1\right\vert}}
    \newcommand{\attention}{\raisebox{-1pt}{{\makebox[1.6em][c]{\makebox[0pt][c]{\raisebox{.13em}{\small!}}\makebox[0pt][c]{\color{red}\Large$\bigtriangleup$}}}}}
    \DeclareMathOperator{\meq}{\stackrel{!}{=}}
    
%..GRAPHICS
	\newcommand{\cgraphic}[2]{\begin{center}\includegraphics[width=#1\columnwidth,keepaspectratio]{#2}\end{center}}
	\newcommand{\mgraphic}[2]{\begin{wrapfigure}{l}{#1\linewidth}\includegraphics[width=\linewidth]{#2}\end{wrapfigure}}
    
% section color box
\setkomafont{section}{\mysection}
\newcommand{\mysection}[1]{%
    \Large\sffamily\bfseries%
    \setlength{\fboxsep}{0cm}%already boxed
    \colorbox{\StyleColor!40}{%
        \begin{minipage}{\linewidth}%
            \vspace*{2pt}%Space before
            #1
            \vspace*{-1pt}%Space after
        \end{minipage}%
    }}

%subsection color box
\setkomafont{subsection}{\mysubsection}
\newcommand{\mysubsection}[1]{%
    \normalsize \sffamily\bfseries%
    \setlength{\fboxsep}{0cm}%already boxed
    \colorbox{\StyleColor!20}{%
        \begin{minipage}{\linewidth}%
            \vspace*{2pt}%Space before
             #1
            \vspace*{-1pt}%Space after
        \end{minipage}%
    }}

% highlighter
\newcommand{\hilight}[1]{\colorbox{\StyleColor}{#1}}
\newcommand{\highlighty}[1]{%
  \setlength{\fboxsep}{0pt}\colorbox{yellow!100}{\ensuremath{#1}}}

\newcommand{\highlightg}[1]{%
  \setlength{\fboxsep}{0pt}\colorbox{green!100}{\ensuremath{#1}}}

\newcommand{\highlightbg}[1]{%
   \colorbox{green!100}{$\displaystyle #1$}}  

% equation box        
\newcommand{\eqbox}[1]{\setlength{\fboxrule}{1mm}\fcolorbox{\StyleColor}{white}{\hspace{0.5em}$\displaystyle#1$\hspace{0.5em}}}

%center equationbox
\newcommand{\ceqbox}[1]{\vspace*{4pt} \begin{center}\eqbox{#1}\end{center}\vspace*{4pt}}


% \bibliography{semiconductordevices}
% \bibliographystyle{ieeetr}

%change page style for header
\pagestyle{fancy}
\footskip 20pt

% Uncomment this line to make formulasheet ultra compact
% This removes
% - list of variables
% \newcommand{\makeultracompact}{irrelevant}
\let\makeultracompact\undefined

% Make stuff ultra compact if so desired
\ifdefined\makeultracompact
  \setlength{\parskip}{0pt}
  \setlength{\abovedisplayskip}{0pt}
  \setlength{\belowdisplayskip}{0pt}
  \setlength{\abovedisplayshortskip}{0pt}
  \setlength{\belowdisplayshortskip}{0pt}
\else
\fi
 
% -----------------------------------------------------------------------
\IfFileExists{../build/revision.tex}{
  \input{../build/revision.tex}
  \rhead{Compiled: \compiledate \hspace{1em} on: \hostname \hspace{1em} from commit: \revision \hspace{1em} Noah Huetter}
}{\rhead{Noah Huetter}}

\ifdefined\makeultracompact
  \lhead{ETH Semiconductor Devices 2019 \hspace{1em}compact version}
\else
  \lhead{ETH Semiconductor Devices 2019}
\fi
\chead{\thepage}
\cfoot{}
\headheight 17pt \headsep 10pt
\title{ETH Semiconductor Devices 2019}
\author{Noah Huetter}

\date{\today}
\begin{document}

\setcounter{page}{0}
\setcounter{secnumdepth}{2} %no enumeration of sections
\begin{multicols*}{4}
	\section*{Disclaimer}
	This summary is part of the lecture ``ETH Semiconductor Devices'' by Prof. Dr. Colombo Bolognesi (FS19). It is based on the lecture. \\[6pt]
	Please report errors to \href{mailto:huettern@student.ethz.ch}{huettern@student.ethz.ch} such that others can benefit as well.\\[6pt]	
  The upstream repository can be found at \href{https://github.com/noah95/formulasheets}{https://github.com/noah95/formulasheets}
	\vfill\null
  \columnbreak
  %%%%%%%%%%%%%%%%%%%%%%%%%%%%%
  \tableofcontents
  \vfill\null
  %\columnbreak
  %%%%%%%%%%%%%%%%%%%%%%%%%%%%%
	\pagebreak
  \maketitle 
  \setcounter{page}{1}
  \thispagestyle{fancy}

  % ---------------------------------------------------------------------------
  \section{Introduction}
  % ---------------------------------------------------------------------------
    \subsection{Electric resistivity/conductivity}
    \cgraphic{0.8}{img/conductor.png}
    \ifdefined\makeultracompact\else
      Conductivity $\sigma$ is a material property describing how easily certain material can conduct electrical current.
      Resistivity $\rho = 1/\sigma$ describes how much a material opposes the current flow.
      The resistance of a square/round piece of metal is:
      \cgraphic{0.8}{img/resistivity.png}
    \fi
    \[R=\rho\frac{l}{A}=\rho\frac{l}{r^2\pi}\]

    \subsection{Electron motion}
    \cgraphic{0.5}{img/electronmotion.png}
    \begin{itemize}
      \item Electric Force Definition
        \[\vec{F}_e = Q\vec{E} \quad \vec{E}:\frac{\text{Newtons}}{C}=\frac{V}{m}\]
      \item Magneti Force Definition:
        \[\vec{F}_m = Q\vec{v}\times\vec{B} \quad \text{not used in course}\]
      \item Definition of electric field:
        \[F_{12}=F_{21}=k\frac{Q_1Q_2}{r^2} \quad E_{21}=\frac{F_21}{Q_2}=k\frac{Q_1}{r^2}\]
      \item Definition: Current
        \[I = \abl{t}Q\]
    \end{itemize}

    \subsection{Current flow}
    \cgraphic{0.5}{img/current.png}
    
    \subsection{Moore's Law}
    Gordon Moore predicted that the number of transistors on an integrated circuit doubles about every two years. This is described using exponential growth:
    \[p(t) = p_0 \cdot b^{t/\tau}\]
    Where
    \begin{align*}
      p(t) &= \text{population at given time} \\
      p_0 &= \text{initial population} \\
      b &= \text{growth rate per time constant} \\
      \tau &= \text{time constant}
    \end{align*}

  % ---------------------------------------------------------------------------
  \section{Solid state physics}
  % ---------------------------------------------------------------------------
  \subsection{Crystal structures}
  \cgraphic{0.3}{img/wholecrystal.png}
  \textbf{Coordination Number}: Is the number of nearest neighbours any atom has in a given crystal lattice. By definition, a crystal lattice is periodic in 3D. $a$ is the lattice constant.

  \cgraphic{1}{img/crystalstructures.png}
  \begin{tabular}[h]{l l p{1.5cm}}
    (a) & SC: Simple cubic & Po \\
    (b) & BCC: Body-centered cubic & Li, Na, K, Cr, Fe, NB \\
    (c) & FCC: Facae-centered cubic & Al, Ar, Ni, Cu, Kr, Pd \\
  \end{tabular}

  \subsection{Silicon}
  \cgraphic{0.7}{img/siliconstructure.png}
  \textbf{Diamond unit cell}: (a) the cubic unit cell, and (b) the inherent tetrahedral structure. The diamon crystal structure is epecially important in semiconductors. For Silicon, the nuclear diameter is $7.2\text{fm} = 2.7\cdot 10^{-6}\text{nm}$: Matter is impressively ``empty''.
  \[a = \frac{4}{\sqrt{3}}  \cdot \text{nearest neighbour distance} \]
  
  \subsection{Crystal Planes and Directions}
  \cgraphic{0.8}{img/planes.png}
  \ifdefined\makeultracompact\else
    Things don't look the same in all directions. 
    The crystal has different periodicities in different directions, \textit{i.e.} \textbf{it does not look the same in all directions}. 
    Different crystal planes and directions generally have different properties.
  \fi
  
  \subsubsection{Miller indices}
  \[[abc] = \left[\frac{1}{p},\frac{1}{q},\frac{1}{r}\right]\]
  Where $p,q,r$ are the intersections with the $x,y,z$ axis. Miller indices describe the crystal plane.
  
  \ifdefined\makeultracompact\else
    \subsection{Elements}
    \cgraphic{1}{img/elements.png}
    \textbf{Electronegativity}: Tendency to attract electrons. Increases from Bottom to Top / Left to Right.
  \fi

  % \subsubsection{Simple Metals}
  % (\textit{e.g.} Alkali Metals, like Na) Collective interaction of mobile electron fluid with positive metal ions. 
  % \textbf{Occurs when teh coordination number is greater than the nu,ber of valenve electrons}. Close packed structures. The material is mechanically soft. 
  % \subsubsection{Transition Metals}
  % (\textit{e.g.} Mo, W) Bon is covalent-like. Transision metals are much harder than simple metals and located in the center of the transition metals row.
  % \subsubsection{Transition Metals}
  % (Column \rom{4}, \rom{3}-\rom{5}, \rom{2}-\rom{6}) Based on hybridization of ``s'' and ``p'' orbitals. 
  % \textbf{Bonds are very directional}, material is mechanically stiff (brittle). Generally habe a diamond-like crystal structure.

  \subsection{Band Gap}
  \cgraphic{1}{img/bandgap.png}
  \ifdefined\makeultracompact\else
    In solids, we are concerned about the atoms' outermost (\textit{i.e.} valence) electrons because they determine the bonding and the electronic properties.
    On the far right in the figure, atoms are well seperated and non-interacting. 
    Their energy states are sharp (atomic-like). 
    As they get closer, the outermost (valence) electrons begin to interact and their energy levels start to shift with respect to the isolated value. 
    \textbf{The number of states is conserved}. 
    This is a consequence of the \textbf{Pauli Exclusion Principle}. 
    Bands can be seperated by \textbf{energy gaps} where no electron is permitted to exist.
  
    \subsubsection{Partially Filled / Empty Band}
    Are associated with electrical \textbf{conduction}. 
    As atoms approach each other to form solid, valence electron distributions overlap. 
    Equilibrium distance ~ maximum density of electrons for isolated atoms.  
    Lowering of potential barriers between atoms allows electrons to move freely.

    \subsubsection{Full Band}
    \textbf{Isolation}, electrons are there but no net current. 
    At equilibrium atomic separation, bands are seperated by a forbidden energy gap. 
    At low T, valence band full, conduction band empty: \textit{i.e.} no current.
  \fi

  \subsection{Population of Electron States}
  \ifdefined\makeultracompact\else
    Need to know the probability of finding electrons at a given energy to understand how they are distributed aming the various states 
    (\textit{i.e.} conduction and valence band). 
    This is done by the \textbf{Fermi Dirac Statistics (FD)}. 
    FD statistics enforce the Pauli Exclusion Principle and the monimaztion of energy.
  \fi
  
  Probability of finding $e^-$ at energy $E$ is $F(E)$
  Probability of finding a hole at energy $E$ is $1-F(E)$

  \cgraphic{0.8}{img/fermidirac.png}
  \[F(E) = \frac{1}{1+\exp\left(\frac{E-E_F}{kT}\right)}\]
  \textbf{Fermi Level $E_F$}: Energy at which the prob. of finding an $e^-$/hole is 50\%

  \subsubsection{Density of states}
  \cgraphic{1}{img/densityostates.png}
  \textbf{Density of States} $g(E)$: How many available states per volume at energy $E$.
  
  \subsection{Intrinsic carriers}
  \cgraphic{1}{img/intrinsiccarriers}
  Two types of carriers: electrons and holes.
  
  \ifdefined\makeultracompact\else
    Each Si atom is surrounded by 8 $e^-$ at 0K, 4 come from the atom itself, 1 from each of its 4 nearest neighbours.
    As $T$ increases, thermal energy eventually excites some $e^-$ out of their bond. 
    Equivalently, the FD distribution broadens around $E_F$, increasing the prob. that conduction band states become occupied.
    A empty state (a ``hole'') is left in the bonding (valence band) state. 
    The valence band is then not full anymore, and it therefore becomes conductive too.
    \textbf{Consequence}: We now have two partially filled bands. 
    Both the conduction and valence bands can now carry current because neither is completely full, not empty.
  \fi
  
  \subsubsection{Detailed balance}
  In equilibrium, $n_0$ and $p_0$ are constant but the reaction still operates in both directions. This is called ``detailed balance''.
  \ceqbox{n_0 \cdot p_0 = K = K_0 \exp\left(\frac{-E_a}{kT}\right)= n_i^2}
  Free electron $n_0$ and hole $p_0$ density and activation energy $E_a$ for bond breaking.

  \subsubsection{Intrinsic carrier density}
  \cgraphic{0.8}{img/intrinsiccarrierdensity}
  \ifdefined\makeultracompact\else
    The intrinsic carrier concentration depends exponentially on temperature $T$
    And on the energy gap of the material (Si: 1.12 eV, GaAs: 1.42 eV, @ 300K)
    Different energy gaps thus have a huge impact on $n_i$
    $n_i$ becomes large at high $T$ (limits high-$T$ operation of devices)
  \fi

  \subsubsection{Electric field}
  Electrons move opposite to electric field, holes move parallel to E-field.

  % ---------------------------------------------------------------------------
  \section{Doping}
  % ---------------------------------------------------------------------------
  Doping is the introduction of impurities with different atomic valence in the pure crystal. These impurities are called \textit{Extrinsic Carriers}.


  \ifdefined\makeultracompact
  \else
  \begin{tabular}[h]{l l}
    $E_D$   & Donor energy level/state \\
    $E_c$   & Conduction band edge\\
    $E_v$   & Valence band edge\\
    $E_F$   & Fermi energy\\
    $E_{Fi}$   & Intrinsic Fermi energy\\
    $E_g$   & Band gap width/energy\\
    $m^*$   & Effective mass of electron(n)/hole(p) \\
    $\epsilon_0$ & Vacuum permittivity $8.854\cdot 10^{-12}$F/m\\
    $n_i$   & Intrinsic electron concentration \\
    $p_i$   & Intrinsic hole concentration\\
    {}      & $n_i=p_i$\\
    $n_0$   & Thermal-equilibrium $e^-$ concentration\\
    $p_0$   & Thermal-equilibrium hole concentration\\
    $n_d$   & Concenctraiton of $e^-$ in donor state \\
    $p_a$   & Concenctraiton of holes in acceptor state \\
    $N_d$   & Concentration of donor atoms\\
    $N_a$   & Concentration of acceptor atoms\\
    $N_c$   & Effective density of states\\
    $N_v$   & Effective density of states\\
    $N_d^+$   & Conc. of pos. charged (ionized) donors \\
    $N_a^-$   & Conc. of neg. charged (ionized) acceptors \\
    % $$   &  \\
  \end{tabular}
  \fi

  \subsection{Impurity doping}
  It is distinguished between negative and positive doping.
  
  \subsubsection{N-Doping}
  \cgraphic{0.7}{img/ndope}
  Taking donors from PT col \rom{5} (P, As, Sb) that have 5 valence electrons.
  \ifdefined\makeultracompact\else  
    4 make covalent bond, 5\ts{th} is ``not needed'' and looks like a hydrogen atom.
    It is easily inozed/excited to conduction band. Its energy state is close to CB ($E_c$).
    \attention{Overall, the solid is still charge neutral but with impurities}\attention

    Donors introduce an energy state $E_D$ near the conduction band edge $E_c$.
    $e_-$ easily promoted to conduction band because $E_c$, $E_D$ are close compared to $E_c$, $E_v$ ($E_g$).
    Only $E_c-E_D$ is needed to free the electron. \textbf{Extra electrons are added without adding holes.}
  \fi

  \subsubsection{P-Doping}
  \cgraphic{0.7}{img/pdope}
  Same as n-doping but by adding extra holes. Acceptors from PT col \rom{3} (B, Al, Ga, In)

  \subsubsection{Energy Band}
  \cgraphic{1}{img/dopeband}
    \begin{tabular}[h]{p{0.45\linewidth} | p{0.45\linewidth}}
    \textbf{n-type doping} & \textbf{p-type doping} \\
  \end{tabular}

  \cgraphic{0.8}{img/bandenergies}
  
  \ifdefined\makeultracompact\else
    Image shows impurity levels in Silicon in eV. 
    This amount of energy is required to move impurity to conductance/valence band.
  \fi
  
  \ifdefined\makeultracompact\else
    Donors: treat as ``Hydrogen-Like'' atom inside solid. Its ionization energy $E_D$ is modified by the dielectric constant and effective mass $m^*$.

    \subsubsection{Bohr radius}
    The most probable distance between nucleus and electron in a Hydrogen atom in ground state.
    For semiconductor we need to modify this value for $\epsilon_0$ and $m^*$.
    This sphere corresponds to about 6735  Si atoms or 841 unit cells. Is thus very loosely bound to its ``parent'' impurity atom.
  \fi

  \[E_d\approx \frac{1}{\epsilon_r^2}\frac{m^*}{m_0}E_H\]
  $E_{D,\text{Bohr}} = 27\text{meV}$ a very good approximation.
  \[r'_{\text{Bohr}}=\frac{4\pi\epsilon_r\epsilon_0\hbar^2}{m^*m_0q^2} = \frac{\epsilon_r}{m^*}5.3\cdot10^{-9}=31\angstrom\]

  \subsection{Electro neutrality}
  \ifdefined\makeultracompact\else
    Sum of atoms making up the semiconductor are elctrically neutral: The semiconductor thus has zero net charge.
    In general, both donor and acceptor impurities may be present.
  \fi
  \[n_0 + N_a^- = p_0 + N_d^+\]

  \subsection{Density of States (DOS)}
  \cgraphic{1}{img/dos}
  Density of available quantum states times Fermi-Dirac distribution equals the density of electrons. $D(E_\text{kin})$ is the number of electronic states at energy $E_\text{kin}$ in a range $\delta e_\text{kin}$ per $cm^3$.
  \[D(E_\text{kin}) = \frac{8}{h^3}\sqrt{2}\pi(m^*)^{3/2}(E_\text{kin})^{1/2}\]

  \subsection{Population of Electron States}

  \ifdefined\makeultracompact\else
    Concentration of electrons and holes ($g(E)$ int Neamen p. 109) is sum of all density of states times Fermi-Dirac probability:
    \[\int_{E=0}^\infty f(E) \cdot D(E_\text{kin})\Dx E_\text{kin}\]
  \fi

  \[f(E) = \frac{1}{1+\exp\left(\frac{E-E_F}{kT}\right)} \approx e^{\frac{E-E_F}{kT}} \quad\text{if} \quad E \gg E_F\]
  \begin{align*}
    n_0 &= \frac{4\sqrt{2}(\pi m^* kT)^{2/3}}{h^3}e^{-(E_c-E_F)/(kT)} \\
        &= N_c \cdot e^{-(E_c-E_F)/(kT)}
  \end{align*}

  For $E \gg E_F$ the Maxwell-Boltzmann approximation holds.

  \subsubsection{General case}
  \cgraphic{1}{img/dosgeneralcase}
  \ifdefined\makeultracompact\else
    When the Fermi level $E_F$ is more than $3kT$ below $E_c$ (or above $E_v$), the full Fermi-Dirac (FD) integral $F_{1/2}(\eta)$ is well approximated by the Maxwell-Boltzmann (MB) approximation $e^\eta$.
    When $E_F$ is colser or even above $E_c$ (below $E_v$), the material is said to be ``degenerate''.
  \fi

  \subsubsection{Concentration of electrons and holes}
  Back to (MB).
  \begin{align*}
    n_0 &= N_c \cdot e^{-\frac{E_c-E_F}{kT}} & N_c &= \frac{4\sqrt{2}(\pi m^*_n kT)^{3/2}}{h^3} \\
    p_0 &= N_v \cdot e^{-\frac{E_F-E_v}{kT}} & N_v &= \frac{4\sqrt{2}(\pi m^*_p kT)^{3/2}}{h^3} 
  \end{align*}

  We see that the intrinsic carrier concentration depends on the temperature and the energy gap of the semiconductor:
  \[n_0 p_0 = N_v N_c \cdot e^{-\frac{E_c-E_v}{kT}} = n_i^2 = N_vN_c\cdot e^{-\frac{E_g}{kT}}\]

  Remember: \ceqbox{n_0p_0 = n_i^2}
  
  \subsection{Book Chapter 4 Summary}

      \begin{tabular}[h]{l c c c}
        Material & Si & Ge & GaAs \\
        $n_i^2(cm^{-6})$  & $9.3\cdot 10^{19}$  & $5.76\cdot 10^{26}$ & $3.24\cdot 10^{12}$ \\
        $N_c(cm^{-3})$    & $2.86\cdot 10^{19}$ & $1.04\cdot 10^{19}$ & $4.7\cdot 10^{17}$ \\

        $N_v(cm^{-3})$    & $1.04\cdot 10^{19}$ & $6.0\cdot 10^{18}$  & $7.0\cdot 10^{18}$ \\
        $E_g (eV)$        & $1.12$ & $0.66$ & $1.42$ \\
        $m_n^*/m_0$       & $1.08$ & $0.067$ & $0.55$\\
        $m_p^*/m_0$       & $0.56$ & $0.48$ & $0.37$\\
      \end{tabular}

    \textbf{Intrinsic Band Gap} \\
      \begin{align*}
        E_g &= -kT\ln\frac{n_i^2}{N_c N_v} &
        n_i^2 &= N_c N_v e^{-\frac{E_g}{kT}} \\
        n_0 &= N_ce^{\frac{-(E_c-E_F)}{kT}} &
        p_0 &= N_ve^{\frac{-(E_F-E_v)}{kT}}
      \end{align*}

    \textbf{Effective Mass} \\
      \begin{align*}
        N_c &= 2\left(\frac{2\pi m_n^*kT}{h^2}\right)^{3/2} &
        N_v &= 2\left(\frac{2\pi m_p^*kT}{h^2}\right)^{3/2} &
      \end{align*}

    \textbf{Intrinsic Carrier Concentration} \\
      \begin{align*}
        n_i^2 &= N_cN_ve^{\frac{-(E_c-E_v)}{kT}} = N_cN_ve^{\frac{-E_g}{kT}}
      \end{align*}

    \textbf{Intrinsic Fermi Level} \\
      \begin{align*}
        E_{Fi} - E_{\text{midgap}} &= \frac{3}{4}kT\ln\left(\frac{m_p^*}{m_n^*}\right) \\
        E_{Fi} &= \frac{E_v+E_c}{2} + \frac{kT}{2}\ln\left(\frac{N_v}{N_c}\right)
      \end{align*}

    \textbf{Equilibrium distribution of el/holes} \\
      \begin{align*}
        n_0 &= n_i e^{\frac{E_F-E_{Fi}}{kT}} &
        p_0 &= n_i e^{\frac{-(E_F-E_{Fi})}{kT}} &
      \end{align*}

    \textbf{Statistics of Donors and Acceptors} \\
      \begin{align*}
        n_d &= \frac{N_d}{1+\frac{1}{2}e^{\frac{E_d-E_F}{kT}}} &
        p_d &= \frac{N_a}{1+\frac{1}{2}e^{\frac{E_F-E_a}{kT}}} \\
      \end{align*}

    \textbf{Thermal-Equilibrium El. Concentration} \\
    $n_0$ for $N_d>N_a$ (n-type), $p_0$ for $N_a>N_d$ (p-type).
      \begin{align*}
        n_0 & = \frac{N_d-N_a}{2}+\sqrt{\left(\frac{N_d-N_a}{2}\right)^2+n_i^2} \\
        p_0 & = \frac{N_a-N_d}{2}+\sqrt{\left(\frac{N_a-N_d}{2}\right)^2+n_i^2} &
      \end{align*}
 
    \textbf{Position of Fermi Energy Level} \\
    Use the $n_0$ formula for n-type, the $p_0$ formula for p-type.
      \begin{align*}
        E_F-E_{Fi} &= kT\ln\left(\frac{n_0}{n_i}\right) &
        E_{Fi}-E_{F} &= kT\ln\left(\frac{p_0}{n_i}\right) &
      \end{align*}

  % ---------------------------------------------------------------------------
  \section{Excess Carriers}
  % ---------------------------------------------------------------------------
  \cgraphic{0.9}{img/dopingvars.png}

  \ifdefined\makeultracompact
  \else
    \begin{tabular}[h]{l l}
      $G_{th}$   & Thermal generation rate \\
      $R_{th}$   & Thermal recombination rate \\
      $G_{L}$    & Photonic generation rate \\
      $\tau_p, \tau_n$   & Minority carrier lifetime \\
      $\Delta_p, \Delta_n$   & Excess carrier concentration to equi. \\
      $N_t$    & Density of recombination centers \\
      $\sigma$    & Recomb. center cross section \\
      {}    & \textbf{or} conductivity \\
      $v_{th}$    & Carrier mobility in thermal equilibrium \\
      $R_a,R_b$    & El. indirect capture and emission rate \\
      $R_c,R_d$    & Hole indirect capture and emission rate \\
      $e_n,e_p$    & El./Hole indirect emission probability \\
      $U$    & Net recombination rate \\
      $J_{\text{diff}}$ & Diffusion current dens. \\
      $D$ & Diffusion constant \\
      $q$ & electron charge \\
      $J_n$ & Electron diffusion current dens.\\
      $v_{dr,n/p}$ & Electron/hole drift velocity \\
      $J_{dr,n/p}$ & Electron/hole drift current dens.\\
      $\mu_{n/p}$ & Electron/hole mobility \\
      $J_{n/p}$ & Electron/hole total current dens. \\
    \end{tabular}
  \fi

  \subsection{Direct Generation/Recombination}
    \cgraphic{0.8}{img/genrecomb.png}

    \ifdefined\makeultracompact\else
      Thermal (spontaneous) and external generation (e.g. light) accross the energy gap.

      In steady state, at a given T: Electrons are continually generated due to thermal energy.
      Some electrons recombine with holes, so that on average $n_0$ and $p_0$ are constant. 
      Rate is proportional to $n_0 \cdot p_0$. 
    \fi
    \[G_{th} = R_{th} = \beta(n_0 \cdot p_0) = \beta n_i^2\]

    This is because we always need 1 electron and 1 hole to "meet" for each recombination event.
    \begin{align*}
      R &= \beta n p \quad G = G_L + G_{th} \quad \Delta n = \Delta p\\
      R &= \beta\cdot n_n p_n = \beta(n_{n0}+\Delta n)(p_{n0} + \Delta p)  \\
      \abl{t}p_n &= G-R
    \end{align*}

    Mass action law ($n_0p_0=n_i^2$) states for n-type doping that:
    \[p_{n0} \ll n_{n0}\]

    Low level injection is when
    \[\Delta p \ll n_{n0}\]

    Then the net recombination rate can be written as:
    \[U = \beta(n_{n0}+p_{n0}+\Delta p)\Delta p \approx \beta n_{n0}\Delta p = \frac{\Delta p}{\tau_p}\]

    \ifdefined\makeultracompact\else
      The minority carrier lifetime describes how fast the excess carrier concentration decays towards equil. when excitation ends. 
      Determined by majority carrier concentration.
    \fi
    \includegraphics[width=0.7\columnwidth,keepaspectratio]{img/lightonoff.png}\eqbox{\tau_p = \frac{1}{\beta n_{n0}}}

    Light on:
    \begin{align*}
      G_L &= U = \frac{p_n-p_{n0}}{\tau_p} & p_n(t \leq 0) &= p_{n0} + \tau_pG_L
    \end{align*}

    Light off:
    \begin{align*}
      \Abl{p_n}{t} &= G_{th} - R = -U = -\frac{p_n-p_{n0}}{\tau_p} \\
      p_n(t) &= p_{n0} + \tau_p G_L \exp\left(-\frac{t}{\tau_p}\right)
    \end{align*}

    Generally:
    \eqbox{\Delta n = \tau_{n_0}R_n}

  \subsection{Indirect Generation/Recombination}
    \cgraphic{0.7}{img/trap.png}
    Energy trap near midgap.
    \[ U \approx  \frac{v_{th}\sigma_0N_t\cdot\Delta p}{1 + \frac{2n_i}{n_{n0}}\cosh\frac{E_t - E_i}{kT}} \approx \frac{\Delta p}{\tau_p} \approx \frac{p_n - p_{n0}}{\tau_p}\]
    Where $N_tv_{th}\sigma$ are the recombination events taking place per unit time, $v_{th}\sigma$ a cylindrical colume in material per unit time.
    \[\frac{1}{2}m_nv_{th}^2=\frac{3}{2}kT \quad v_{th}\approx 10^7 cm\cdot s^{-1}\]
    Electron capture ($R_a$) and emission ($R_b$) rate must be equal in therm. equi.
    \[R_a = nN_t(1-f)\cdot v_{th}\sigma_n \quad R_b = e_nN_tf\]
    The emission probability increases exponentially as $E_t$ gets closer to conduction band edge:
    \[e_n = \frac{v_{th}\sigma_nn(1-f)}{f}=v_{th}\sigma_nn_ie^{(E_t-E_i)/kT}\]
    For holes:
    \[R_c = pN_tf\cdot v_{th}\sigma_p \quad R_d = e_pN_t(1-f)\]
    \[e_p = v_{th}\sigma_p n_i e^{(E_i-E_t)/kT}\]
    These lead to the equation for $U=\dots$.

  \subsection{Diffusion}
    \ifdefined\makeultracompact\else
      Result of concentration gradients.
      Equal probability of moving in any direction.
      Fick's first law of diffusion:
    \fi
    \[J_{\text{diff}} = -D\Delta N = -D \left( \Pabl{N}{x}\vec{x_u} + \Pabl{N}{y}\vec{y_u} \dots \right)\]
    \ifdefined\makeultracompact\else
      In thermal equi. and uniform distribution, free charge carriers are in constant motion. 
      Net current is thus zero.
      Statistical mechanics show that particles at temp. $T$ have avg. thermal energy of $3kT/2$
      For a particle of mass $m$ this corresponds to an avg. thermal velocity
    \fi
    \[\frac{1}{2}mv_{th}^2=\frac{3}{2}kT \quad J_{\text{diff},n} = -qF = qD_n\Abl{n}{x}\]
    \ifdefined\makeultracompact\else
      In a crystal we must use the \textbf{effective mass}
      The electron diffusion current:
    \fi

  \subsection{Drift}
    \ifdefined\makeultracompact\else
      Result of an electric field as driving force.
      Zero field: Electrons move thermally (randomly) in all directions (no net flow)
      Non-zero field: There is a net drift of electrons, opposite to the E-field
      We can then define a drift velocity of electrons $v_{dr,n}$
      Electrons do not accelerate indefinitely due to collisions
    \fi
    \begin{align*}
      J_{dr,n} &= -qnv_{dr,n} & v_{dr,n} &= -\mu_nE \\
      J_{dr,p} &= qpv_{dr,h} & v_{dr,h} &= \mu_pE \\
      J_{dr,tot} &= \sigma E & \sigma &= q(n\mu_n + p\mu_p)
    \end{align*}
  
  \subsection{Total current}
    Total current = drift + diffusion current = electron + hole current.
    \begin{align*}
      &\text{Electrons:} & J_n &= nq\mu\vec{E}+qD_n\Abl{n}{x} \\
      &\text{Holes:} & J_p &= \underbrace{pq\mu\vec{E}}_\text{drift}\underbrace{-qD_p\Abl{p}{x}}_\text{diffusion} \\
    \end{align*}

  % ---------------------------------------------------------------------------
  \section{Excess Carriers}
  % ---------------------------------------------------------------------------
   
  \ifdefined\makeultracompact
  \else
    \begin{tabular}[h]{l l}
      $G_{n/p}$   & Generation rate of el/hole \\
      $R_{n/p}$   & Recombination rate of el/hole \\
      $J_{n/p}$   & Current density of el/hole \\
      $L_{n/p}$   & Minority carrier diffusion length \\
      $D_{n/p}$   & Diffusion constant \\
      $\mu_{n/p}$ & Carrier mobility \\
    \end{tabular}
  \fi

  \subsection{Continuity equation}
    \textbf{Objective:} Accounting for carrier densities when drift, diffusion and G/R take place.
    \cgraphic{0.5}{img/continuity.png}
    
    \begin{align*}
      &\text{Change in e} & \Pabl{n}{t} &= \frac{1}{q}\Pabl{J_n}{x}+(G_m-R_n) \\
      &\text{Change in hole} & \Pabl{p}{t} &= -\frac{1}{q}\Pabl{J_p}{x}+(G_p-R_p)
    \end{align*}

    Written for \textbf{electrons} in p-type material
    \begin{align*}
      \Pabl{n_p}{t} &= n_p\mu_n\Pabl{\vec{E}}{x}+\mu_n\vec{E}\Pabl{n_p}{x} + D_n\Pablq{n_p}{x} \\ &+ G_n - \frac{n_p-n_{p0}}{\tau_n}
    \end{align*}

    Written for \textbf{holes} in n-type material
    \begin{align*}
      \Pabl{p_n}{t} &= p_n\mu_p\Pabl{\vec{E}}{x}+\mu_p\vec{E}\Pabl{p_n}{x} + D_p\Pablq{p_n}{x} \\ &+ G_p - \frac{p_n-p_{n0}}{\tau_p}
    \end{align*}

    \subsubsection{Steady state}
    For steady state considerations:

    \ceqbox{0=D_n\Pablq{n_p}{x}+G_n-\frac{n_p-n_{p0}}{\tau_n}}
    
    \subsubsection{Einstein relation}
    \begin{center}
      \eqbox{D_p = \frac{kT}{q}\mu_p} \eqbox{D_n = \frac{kT}{q}\mu_n}
    \end{center}

    \subsubsection{Semi-infinite sample}
    \cgraphicd{0.49}{img/semiinf1.png}{img/semiinf2.png}
    \begin{align*}
      p_n(x) &= p_{n0} + (p_n(0)-p_{n0}) \exp\left(-\frac{x}{L_p}\right) \\
      L_p &= \sqrt{D_p\tau_p}
    \end{align*}

    \subsubsection{Finite sample}
    \cgraphicd{0.49}{img/finitesample1.png}{img/finitesample2.png}
    \begin{align*}
      p_n(x) &= p_{n0} + (p_n(0)-p_{n0}) \frac{\sinh\left(\frac{W-x}{L_p}\right)}{\sinh\left(\frac{W}{L_p}\right)}
    \end{align*}
    
    \subsubsection{Short diode}
    If $W \ll L_p$ then recombination is weak, few holes recombine in the time required for them to cross the region $W$.
    The excess carrier profile bebomes linear:
    \[p_n(x) = p_{n0} + (p_n(0)-p_{n0})\frac{W-x}{W}\]

  \subsection{Equilibrium: constant fermi level}
  \cgraphic{0.5}{img/pnfermilevel.png}
  In equilibrium, the Fermi level mus be constant to balance transfer rates so that no net current flows.
  \ceqbox{E_{F1} = E_{F2}}


  \subsection{Doping: Band bending}
  Near the transition region, the distance between the Fermi level and conducition/valence band edge changes.
  Far away from the junction, the material does not ``know'' there is a junction.

  % ---------------------------------------------------------------------------
  \section{PN Junction}
  % ---------------------------------------------------------------------------
  \ifdefined\makeultracompact
  \else
    \begin{tabular}[h]{l l}
      $N_a$   & Acceptor concentration in p-region \\
      $N_d$   & Donor concentration in n-region \\
      $n_{n0}=N_d$   & Equi. maj. carrier e in n reg. \\
      $p_{p0}=N_a$   & Equi. maj. carrier h in p reg. \\
      $n_{p0}=n_i^2/N_a$  & Equi. min. carrier e in p reg.  \\
      $p_{n0}=n_i^2/N_d$  & Equi. min. carrier h in n reg.  \\
      $n_p$   &  Total min. carrier e in p reg.\\
      $p_n$   &  Total min. carrier h in n reg.\\
      $n_p(-x_p)$   &  Min. carr. e in p-reg at depl-edge\\
      $p_n(x_n)$    &  Min. carr. h in n-reg at depl-edge\\
      $\Delta n_p$   & Excess min. car. e in p-reg \\
      $\Delta n_n$   & Excess min. car. h in n-reg \\
      $V_{bi}$& Builtin voltage at equilibrium \\
      $x_n$   & End of depletion region on n-side\\
      $x_p$   & Start of depletion region\\
      $W$     & Depletion region width\\
      $V_a$     & Forward applied voltage\\
      $J_s$     & Reverse saturation cur. dens.\\
    \end{tabular}
  \fi
  \begin{align*}
    \Delta n_p &= n_p - n_{p0} & \Delta n_n &= p_n - p_{n0} \\
  \end{align*}

  \subsection{Diode}
    \cgraphic{0.999}{img/diode}
    The amount of band bending balances the drift/diffusion currents.

  \subsection{Equilibrium}
    \cgraphic{0.7}{img/diode_eq_band}
    \[V_{bi} = \frac{kT}{q}\ln\left(\frac{N_aN_d}{n_i^2}\right)\]
  
  \subsection{Space Charge Width}
    \cgraphic{0.6}{img/space_charge_width}
    \begin{align*}
      x_n &= \left\{ \frac{2\epsilon_sV_{bi}}{q}\left[\frac{N_a}{N_d}\right]\left[\frac{1}{N_a+N_d}\right] \right\}^{1/2}\\
      x_p &= \left\{ \frac{2\epsilon_sV_{bi}}{q}\left[\frac{N_d}{N_a}\right]\left[\frac{1}{N_a+N_d}\right] \right\}^{1/2} \\
      W  &= \left\{ \frac{2\epsilon_sV_{bi}}{q}\left[ \frac{N_a+N_d}{N_aN_d} \right] \right\}^{1/2}
    \end{align*}

  \subsection{Forward Bias}
    Condition:
    \begin{align*}
      \left. \begin{array}{rl}
        n_p(-x_p) &> n_{p0} \\
        p_n(x_n) &> p_{n0} \\
        \end{array} \right\} \text{forward biased}
    \end{align*} 

    Injection strength:
    \begin{align*}
      \left. \begin{array}{rl}
        n_p &< N_a \\
        p_n &< N_d \\
        \end{array} \right\} \text{low level injection} \\
      \left. \begin{array}{rl}
        n_p &> N_a \\
        p_n &> N_d \\
        \end{array} \right\} \text{high level injection}
    \end{align*} 

    Applying an external forward voltage is called forward bias or minority carrier injection.
    This voltage subtracts from $V_{bi}$.
    Minority carriers at depletion edges increase by factor $\exp(qV_F/kT)$.
    Known as \textbf{Shockley Boundary Conditions}.
    \cgraphic{0.6}{img/shockleyhand}
    \begin{align*}
      n_p(-x_p) &= n_{p0}e^{\left(\frac{qV_a}{kT}\right)} & p_n(x_n) &= p_{n0}e^{\left(\frac{qV_a}{kT}\right)}
    \end{align*}
    \eqbox{n_{p0}=\frac{n_i^2}{N_a}} \eqbox{p_{n0}=\frac{n_i^2}{N_d}}

  \subsection{Ideal Junction Current}
    The ideal current $J$ is calculated by:
    \[ J_s = \left[ \frac{qD_pp_{n0}}{L_p} + \frac{qD_nn_{p0}}{L_n} \right] \quad J = J_s \left[ e^\frac{qV_a}{kT} -1 \right] \]
    
    Recalling
    \begin{align*}
      D_i &= \frac{kT}{q}\mu_i = \frac{L_i^2}{\tau_i} & L_i&=\sqrt{D_i\tau_i} & &i\in\{n,p\}\\
    \end{align*}
    \cgraphic{1}{img/bjtcurrent.png}

    \subsubsection{Short diode}
    For a short diode $L_p,L_n$ become $W_d, W_a$ respecively if $W_{a,d} \ll L_{n,p}$.
    \begin{align*}
      J_s &= \left[ \frac{qD_pp_{n0}}{W_d} + \frac{qD_nn_{p0}}{W_a} \right] \\
    \end{align*}
    \cgraphic{1}{img/bjtcurrentshort.png}

    \subsection{One sided junction}
    If one side of a junction is much more doped than the other:
    \begin{align*}
      N_a &\gg N_d \leadsto p^+n  & N_d &\gg N_a \leadsto n^+p 
    \end{align*}

    \subsection{Energy, field, potential, etc.}
    \[\Abl{\vec{E}}{x}=\frac{\rho}{\epsilon_s} \quad \vec{E}(x) = \frac{1}{q}\Abl{E_V}{x}\]
    \[ \abs{\vec{E}_\text{max}}=\frac{qN_ax_p}{\epsilon_s}=\frac{qN_dx_n}{\epsilon_s}=\frac{2(V_{bi}+V_R)}{W} = \left[\frac{V}{m}\right]\]
    \cgraphic{0.5}{img/diodepotential.png}
    
    Draw the curves using following qualitative eqn. $E_P=E_C$ for $e^-$ and $E_P=-E_C$ for hole.
    \[ V \propto E_F - E_C \quad \vec{E} \propto -\Abl{V}{x} \quad E_K + E_P = \text{const.} \]
    \[ \ln(p) \propto E_i-E_F \quad \ln(n) \propto E_F-E_i \]
    \[ J_{\text{drift}} \propto n\vec{E} \quad J_{\text{diff}} \propto \Abl{n}{x} \]
    \cgraphic{1.0}{img/evejnp.png}

    % ---------------------------------------------------------------------------
    \section{Complementary effects}
    % ---------------------------------------------------------------------------
  \ifdefined\makeultracompact
  \else
    \begin{tabular}[h]{l l}
      $C_j$   & Depletion capacitance \\
      $C_d$   & Diffusion capacitance \\
      $J_g$   & Current due to generation \\
      $J_r$   & Current due to recombination \\
      $J_{\text{rev/fwd}}$ & Non-ideal current densities \\
      $\eta$  & Ideality factor \\
    \end{tabular}
  \fi

    \subsection{Summary}
    \renewcommand{\labelenumi}{\alph{enumi})}
    \setlength{\fboxsep}{1pt}
    \begin{enumerate}[topsep=0pt,itemsep=-1ex,partopsep=1ex,parsep=1ex]
    \item \colorbox[rgb]{0.99,0.90,0.77}{Recombination in depletion region}
    \item \colorbox[rgb]{0.84,0.93,0.81}{Ideal injection}
    \item \colorbox[rgb]{0.75,0.85,0.99}{High-level injection}
    \item \colorbox[rgb]{0.89,0.75,0.93}{Series resistance}
    \item \colorbox[rgb]{0.98,0.79,0.78}{Generation in depletion region}
    \end{enumerate}

    \cgraphic{0.8}{img/nonidealsummary.png}
    
    \subsection{Depletion Capacitance}
    For Silicon $\epsilon_r = 11.9$.
    \begin{align*}
      C_j &= \frac{\epsilon_0\epsilon_r}{W} \\
      \frac{1}{C_j^2} &= \frac{1}{\epsilon_0\epsilon_r} \frac{2(V_{bi}-V_f)}{q} \left( \frac{1}{N_a} + \frac{1}{N_d}\right) \\
    \end{align*}

    \subsection{Diffusion Capacitance}
    The stored charged $Q_p$ due to the excess hole density contributes a capacitance $C_d$.
    \begin{align*}
      J &= qD_p\Abl{p_n}{x} = qD_p\frac{p_{n0}\left(e^{qV_f/kT}-1\right)}{W_n} \\
      Q_p &= qW_n\frac{p_{n0}\left(e^{qV_f/kT}-1\right)}{2} \\
      C_d &= \Abl{Q_p}{V_f} = \frac{\tau}{r_d} = \frac{W_n^2}{2D_p}\frac{J}{kT/q}
    \end{align*}

    $C_j$ dominates in reverse bias, $C_d$ becomes dominant in forward bias due to minority carrier charge storage.

    \cgraphic{0.8}{img/cap.png}

    \subsection{Generation Current in reverse bias}
    $J_g$ adds to the ideal reverse saturation current density $J_s$ due to electron hole pair generation.
    \begin{align*}
      J_g &= \frac{qWn_i}{\tau_g} & J_{\text{rev}} &= J_s + J_g\\
    \end{align*}

    \subsection{Reverse Breakdown}
    Tunneling occurs in thin, highly doped junctions while avalanche multiplication arises in thick, more lightly doped junctions.
    Tunneling results in a sharp breakdown characteristic (Zener diode) and avalanch mult. in a more soft slope.
    Higher doping results in lower breakdown voltage and sharper characteristics.
    \begin{align*}
      N_a &\approx N_d \to \text{Tunneling} & N_a &\neq N_d \to \text{Avalanche}
    \end{align*}
    \cgraphic{0.8}{img/breakdown.png}

    \subsection{Recombination in forward bias}
    In forward bias the $pn$ is greater than $n_i^2$: recombination takes place.
    \begin{align*}
      J_r &= \frac{qWn_i}{2\tau_r}e^{qV_f/2kT} & J_{f\text{wd}} &= J_s + J_r\\
    \end{align*}

    \subsection{Series resistance}
    Ohmic losses occur in the undoped regions at high current which leads to voltage drop.
    \cgraphic{0.5}{img/serres.png}

    \subsection{Ideality factor}
    The factor $\eta$ characterizes the diode forward current ideality and is often called the \textit{ideality factor}.
    From graph with two points $I_{f1,2}, V_{f1,2}$ $\eta$ can be calculated.
    \begin{align*}
      J_{FT} &\propto \exp\left(\frac{qV_F}{\eta\cdot kT}\right) & 
      \frac{I_{f1}}{I_{f2}} &= \frac{e^{ qV_{f1}/\eta kT }} {e^{ qV_{f2}/\eta kT }} \to\eta=\dots
    \end{align*}

    % ---------------------------------------------------------------------------
    \section{Depletion approximation}
    % ---------------------------------------------------------------------------
    \subsection{Depletion width correction}
    Using the Gummel correction.
    $2V_t$ for a 2-sided junction, $V_t$ for single sided junction.
    \begin{align*}
      V_t &= \frac{kT}{q} \quad\text{Thermal voltage} \\
      W &= \sqrt{\frac{2\epsilon_s}{q}\left(\frac{1}{N_a} + \frac{1}{N_d}\right)(V_{bi}-2V_t)}
    \end{align*}

    \subsection{Debye Length}
    The length of the region where the doping concentration changes.
    \begin{align*}
      L_D &= \sqrt{\frac{\epsilon_sV_t}{qN_d}}
    \end{align*}

    \subsection{Junction Capacitance}
    $V_t=0$ for depletion approximation.
    \begin{align*}
      \Psi &= V_{bi} + V_R &
      C_j &\approx \sqrt{\frac{\epsilon_sqN_d}{2(\Psi - V_t)}} &
      [C_j] = \frac{F}{m^2}
    \end{align*}

    % ---------------------------------------------------------------------------
    \section{Bipolar transistor}
    % ---------------------------------------------------------------------------
    \subsection{Modes of operation}
    E/B junction injects minority carriers into base. B/C junction extracts minority carriers from base.
    
    \cgraphic{1.0}{img/bjt_op.png}

    \subsubsection{Normal ``Active'' Mode}
    E/B junction fowrard-biased: Minority carrier injection. B/C reverse-biased: Minority carrier extraction.
    Recombination in base consumes e/h pairs.
    
    NPN transistor in \rom{1} equilibrium and \rom{2} forward active mode.
    \cgraphic{0.8}{img/npnbands.png}

    PNP transistor in \rom{1} equilibrium and \rom{2} forward active mode.
    \cgraphic{0.8}{img/pnpbands.png}

    \subsection{Current flow and gain}
    For pnp flip signs and change $n \leftrightarrow p$.
    \cgraphic{0.8}{img/npncurrent.png}
    \[I_B = I_E - I_C = I_{E,p} + (I_{E,n} - I_{C,n}) - I_{C,p}\]

    Emitter efficiency $\gamma$ and base transport factor $\alpha_T$:
    \[ \gamma = \frac{I_{E,p}}{I_{E,p} + I_{E,n}} \quad \alpha_T = \frac{I_{C,p}}{I_{E,p}} \]

    Common emitter current gain in forward- and reverse-active mode:
    \[\beta_F = \frac{I_C}{I_B} = \frac{D_BN_EW_E}{D_EN_BW_B}\frac{A_C}{A_E}\]
    \[\beta_R = \frac{I_E}{I_B} = \frac{D_BN_CW_C}{D_CN_BW_B}\frac{A_E}{A_C}\]
    
    Common base current gain:
    \[\alpha = \frac{\beta}{\beta + 1} \]


    \subsection{Doping}
    Focus is on minority carriers. NPN has pnp minority carriers and PNP npn structure. In this image the BJT is in reverse active mode.
    \cgraphic{0.8}{img/pnpdoping.png}
    \[\frac{\Delta n_E}{n_{E,0}} = \frac{n_E - n_{E,0}}{n_{E,0}} = e^{\frac{qV_{BE}}{kT}} - 1\]
    \[\frac{\Delta p_B}{p_{B,0}} = \frac{p_B - p_{B,0}}{p_{B,0}} = e^{\frac{qV_{CB}}{kT}} - 1\]

    In order to be useful as a transistor we want
    \[ I_{pC} = I_{pE} \gg I_{nE} = I_B \]
    \[ \frac{n_{iB}^2}{N_{DB}} \gg \frac{n_{iE}^2}{N_{AE}} \to N_{AE} \gg N_{DB} \]

    \[ J_C = \frac{qD_{B}}{W_B} \frac{n_i^2}{N_{B}} (e^{qV_{EB}/kT} - 1) = I_S(e^{qV_{EB}/kT} - 1) \]
    \[ g_M = \Abl{I_C}{V_{EB}} = I_S e^{qV_{EB}/kT} \frac{q}{kT} \approx \frac{I_C}{kT/q} \]

    \subsection{Early Effect}
    Collector-emitter voltage decreases base width which increases collector current.
    \cgraphic{0.8}{img/early.png}
    Calculate base width $x_B$ at zero voltage and at two operating points with depletion correction.
    \[W_{BE} = \sqrt{\frac{2\epsilon_i}{q}\frac{N_B+N_E}{N_BN_E}V_x} \quad V_x = V_{bi}-V_{BE} \]
    \[W_{BC} = \sqrt{\frac{2\epsilon_i}{q}\frac{N_B+N_C}{N_BN_C}V_x} \quad V_x = V_{bi}+V_{BC} \]
    \[V_{bi} = \frac{kT}{q}\ln\left(\frac{N_BN_i}{n_i^2}\right) \quad i=E, C\]
    \[x_B = x_{B0} + W_{B,BC,0} + W_{B,BE,0} - W_{B,BC,1} - W_{B,BE,1} \]
    Then calculate collector currents at two operating points.
    \[J_C = \frac{qn_i^2D_B}{N_Bx_B}\left(e^{qV_{EB}/kT}-1\right) \to V_{CE} = V_{BE} + V_{CE}\]
    \[V_A = J_{C,1}\frac{V_{CE,2}-V_{CE,1}}{J_{C,2}-J_{C,1}} - V_{CE,1} \quad \Abl{I_C}{V_{EC}} = \frac{I_C}{V_A + V_{EC}} \]
    \cgraphic{0.8}{img/basewidth.png}
    
    \subsection{Diffused Resistor}

    \cgraphic{0.4}{img/difres.png}
    Resistance is per unit square.
    \begin{align*}
      R_{tot} &= R_\Box\frac{L}{W} + 2R_cR_\Box & R_\Box &= \frac{\rho}{x} \\
      \rho &= \frac{1}{qp\mu_p},\quad p=N_a
    \end{align*}

    \subsection{BJT vs. Back to back diode}
    \cgraphic{0.6}{img/backbackdiode.png}
    BJT: 
    \[I_C = \beta I_B = \frac{\beta}{1+\beta}I_E \quad I_E = I_S\left(e^{qV_{BE}/kT}-1\right)\]
    Diodes:
    \[I_C = I_E - I_B, I_C = I_S \quad I_B = I_S\left(e^{qV_{BE}/kT}-1\right)\]
    


    % ---------------------------------------------------------------------------
    \section{MOS Transistor}
    % ---------------------------------------------------------------------------
    \ifdefined\makeultracompact
    \else
      \begin{tabular}[h]{l l}
        $\phi$   & Work function \\
        $\altchi$    & Electron affinity \\
        $\epsilon_{Ox}$    & $=\epsilon_{0}\epsilon_{r,Ox}$ \\
        $\epsilon_{s}$     &  $=\epsilon_{0}\epsilon_{r,s}$ \\
        $\Psi_B$  & Bulk potential \\
        $\Psi_S$    & Si surface potential  \\
        $\phi_{ms}$    & Work function difference $\phi_m-\phi_s$  \\
        $C_0$  & Oxide capacitance \\
      \end{tabular}
    \fi

    \subsection{Band structure}
    Workfunction $\phi$ is energy required to extract electron. 
    In semiconductor materials the electrons get extraced from conduction band, so we use
    the electron affinity $\altchi$ parameter.
    \cgraphic{1.0}{img/mosband.png}
    \begin{align*}
      &q\phi_s = q\altchi_s + E_g/2 + \phi_{fn} & q\altchi_{SiO2} = q\altchi_s - \Delta E_c \\
      &\phi_{fn} = (E_i - E_F) = kT\ln\frac{N_a}{n_i}
    \end{align*}

    \subsection{Gate votlage dependance}
    The gate voltage creates a depletion region or channel at the interface between Si and oxide depending on the regime.
    The regimes are called \Roe{1} accumulation, \Roe{2} depletion and \Roe{3} inversion mode.
    The \textbf{NMOS} forms a n-channel and therefore uses p-type substrate.
    \cgraphic{1.0}{img/nmosbands.png}
    The \textbf{PMOS} forms a p-channel and therefore uses n-type substrate.
    \cgraphic{1.0}{img/pmosbands.png}

    \textbf{Inversion} 
    Surface potential is double the bulk potential.
    \[\Psi_S = 2\cdot\Psi_B \quad \Psi_B = \frac{kT}{q}\ln\frac{N_a}{n_i}\]

    \textbf{Gate capacitance} 
    in accumulation:
    \[C = C_{Ox} = \frac{\epsilon_{Ox}}{d} \]
    in depletion:
    \[C = (C_{Ox}^{-1}+C_{j}^{-1})^{-1} = \frac{\epsilon_{Ox}}{d + \frac{\epsilon_{Ox}}{\epsilon_S}}W \]

    \textbf{Depletion width} in general $x_d$ and at inversion transision $x_{dt}$:
    \[\phi_{fp} = V_t\ln\frac{N_a}{n_i} \quad x_d = \sqrt{\frac{2\epsilon_s\phi_s}{qN_d}} \quad x_{dt} = \sqrt{\frac{4\epsilon_s\phi_{fp}}{qN_d}}\]
    
    \subsection{Operating mode}
    \begin{align*}
      &\begin{cases}
        V_{DS} \geq V_{GS} - V_t & \text{saturation} \\
        V_{DS}  <   V_{GS} - V_t & \text{linear}
      \end{cases} \\
      I_{D} &= \frac{\mu C_{OX}}{2}\frac{W}{L} (V_{GS} - V_t) ^2 && \text{sat} \\
      I_{D} &= \frac{\mu C_{OX}}{2}\frac{W}{L} \left( 2(V_{GS} - V_t) V_{DS} - V_{DS}^2 \right) && \text{lin} 
    \end{align*}

    \subsection{E-Field}
    Applied voltage $V_a$. If intrinsic, then $\Psi_S = \Psi_B$.
    \[ V_a = \Psi_S + V_{Ox} \quad \Psi_B = \frac{kT}{q}\ln\frac{N_a}{n_i} \quad W=\sqrt{\frac{2\epsilon_s\Psi_S}{qN_a}}\]
    \[ E_S = \frac{qN_a}{\epsilon_s}W \quad E_{Ox}=\frac{E_S\epsilon_s}{\epsilon_{Ox}} \quad V_{Ox} = E_{Ox}d \]

    \subsection{Threshold Voltage}
    Work function difference $\phi_{ms}$ from table.
    \[ \phi_{ms} = \phi_m - \left(\altchi_S + \frac{E_g}{2q} + \frac{E_i-E_F}{q}\right)  \quad C_{Ox} = \frac{\epsilon_{Ox}}{d} \]
    \[ V_t = V_{Ox} + \Psi_S + \phi_{ms} \quad V_{Ox} = \frac{\sqrt{2q\epsilon_s N_a \Psi_S}}{C_{Ox}} \]
    $\Psi_S=2\Psi_B$ if inversion mode.

    \subsection{Frequency dependance}
    Cutoff frequency $f_T$
    \begin{align*} f_T &= \frac{g_M}{2\pi(C_{gs}+C_{gd})}
    = \frac{\mu_n C_{OX} Z(V_{gs} - V_T ) /L }{2\pi\cdot 2C_{OX}ZL/3}  \\
    &= \frac{3}{4}\frac{\mu_n}{\pi}\frac{(V_{gs}-V_T)}{L^2}
    \end{align*}
    
    \subsection{C-V Characteristics}
    \subsubsection{Low gate voltage}
    Accumulation: Only oxide thickness contributes to gate capacitance.
    \subsubsection{Medium gate voltage}
    Depletion: Attracted minority carriers widen the capacitor plates in a sense that the conducting bulk is further away - capacitance decreases.
    \subsubsection{High gate voltage}
    Inversion: At low $f$ a conducting channel is formed and the capacitor plates are closer together meaning higher capacitance.
    At high $f$ the charges in bulk have no time to recombine and the capacitor plates do not move closer resulting in lower capacitance.

    \cgraphic{0.9}{img/moscv.png}
    \[\frac{C}{C_0} = \frac{1}{\sqrt{1+\frac{2\epsilon_{Ox}^2V}{qN_a\epslion_s d^2}}} \quad C_0 = C_{Ox}\]
    
    \subsection{Flatband}
    Flatband condition is met if fermi level through device is constant, this is the case at equilibrium.
    The applied voltage is composed of the oxide portential $V_{Ox}$, the silicon surface potential $\Psi_s$ and 
    the metal-silicon work function difference $\phi_{ms}$.
    \cgraphic{0.9}{img/mosflatband.png}
    
    
    % ---------------------------------------------------------------------------
    % \section{}
    % ---------------------------------------------------------------------------


\end{multicols*}

\setcounter{secnumdepth}{2}
\end{document}
