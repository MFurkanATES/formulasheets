%
% Zusammenfassung Communication Networks D-ITET
% ===========================================================================
% Author:			Marco Dober
% Version:			0.1
% Last changed: 	19.02.2019	
% ---------------------------------------------------------------------------

\documentclass[a4paper, fontsize=8pt, landscape, DIV=1]{scrartcl}
\usepackage{lastpage}
\usepackage{hyperref}
\usepackage[graphicx]{realboxes}
% Include general settings and customized commands
%
% General packages and settings
% ===========================================================================
% Author:			Silvano Cortesi (cortesis@student.ethz.ch)
% Version:			1.2
% Last changed:		03.01.2018
%
% ---------------------------------------------------------------------------




\usepackage[german]{babel} %choose your language \usepackage[german]{babel}
%\usepackage[T1]{fontenc}
\usepackage[utf8]{inputenc}
\usepackage{fancyhdr}
%\usepackage{lastpage}
%\usepackage{lmodern}
\usepackage{enumerate}
%\usepackage{float} % for positioning of figures
\usepackage[landscape, margin=1cm]{geometry}
\usepackage[dvipsnames]{xcolor}
\usepackage{pdfpages}


%% Math %%
\usepackage{todonotes}
\usepackage{amscd}
\usepackage{blindtext}
\usepackage{enumitem}
\usepackage{multicol}
\usepackage{parskip}
\usepackage{empheq}
\usepackage{amsmath}
\usepackage{amsfonts}
\usepackage{amssymb}
\usepackage{amsthm}
%\usepackage{dsfont}
%\usepackage{esint} % provides \oiint
\usepackage{mathrsfs}
%\usepackage{trfsigns}
%\numberwithin{equation}{subsection}
%\usepackage{numprint}

%% Graphics & Charts %%
\usepackage{graphicx}
%\usepackage{pdfpages}
%\usepackage{booktabs}
\usepackage{array}
%\usepackage{paralist}
%\usepackage{framed}
%\usepackage{trfsigns}
\usepackage{tikz}
%\usepackage[lofdepth,lotdepth]{subfig}
%\usepackage{tikz}  %Graphen zeichnen
%\usetikzlibrary{decorations.pathmorphing}
%\usetikzlibrary{arrows.meta,arrows}
%\usepackage{pgfplots}
%% General Settings %%
%\setlength{\parindent}{0px}
%\setkomafont{captionlabel}{\normalfont\bfseries}
\usepackage{wrapfig}
\usepackage{color,soul}

%\pagestyle{fancy}
%\lfoot{\tiny \today}
%\rfoot{\thepage\  / \pageref{LastPage}}
%\cfoot{}
%\renewcommand{\footrulewidth}{0.4pt}

%% provides command \uline{} for underlining words
%\usepackage{ulem}

%% colour headings
%\usepackage{color}
%\definecolor{bluen}{cmyk}{1,0.5,0,0}
%\definecolor{bloodorange}{cmyk}{0,.92,1,.2}
%\addtokomafont{section}{\color{bloodorange}}
%\addtokomafont{subsection}{\color{bloodorange}}
%\addtokomafont{subsubsection}{\color{bloodorange}}
%\addtokomafont{paragraph}{\small\color{bloodorange}}
%\addtokomafont{subparagraph}{\small\color{bloodorange}}

%% Signs & Special Formating %%
%\usepackage{ulem} %normalem: \emph{Text} is italic again.
%\usepackage{multicol,multirow}
%\usepackage{tabularx}
%\usepackage{stackrel}
%\usepackage{makeidx}
%\usepackage{mparhack} % bessere margiale bei seitenumbruch

% make document compact
\usepackage[compact]{titlesec}
\titlespacing{\section}{0pt}{*0}{*0}
\titlespacing{\subsection}{0pt}{*0}{*0}
\titlespacing{\subsubsection}{0pt}{*0}{*0}

\parindent 0pt
\pagestyle{empty}
\setlength{\unitlength}{1cm}
\setlist{leftmargin = *}

%include also newer PDF
% \pdfminorversion=6

% Set the color of your style
% Avaiable are: Apricot, Aquamarine, Bittersweet, Black, Blue, blue, BlueGreen, BlueViolet, BrickRed, Brown, BurntOrange, CadetBlue, CarnationPink, Cerulean, CornflowerBlue, Cyan, Dandelion, DarkOrchid, Emerald, ForestGreen, Fuchsia, Goldenrod, Gray, Green, GreenYellow, JungleGreen, Lavender, ... (more at: http://en.wikibooks.org/wiki/LaTeX/Colors)
\def\StyleColor{MidnightBlue}

%
% General commands
% ===========================================================================
% Author:			Silvano Cortesi (cortesis@student.ethz.ch)
% Version:			1.2
% Last changed:		03.01.2018
%
% ---------------------------------------------------------------------------

%..ROEMISCHE_ZAHLEN
	\newcommand{\Roe}[1]{\uppercase\expandafter{\romannumeral #1 }}

%..ZAHLENMENGEN
	\newcommand{\N}{\mathbb{N}}
	\newcommand{\Z}{\mathbb{Z}}
	\newcommand{\Q}{\mathbb{Q}}
	\newcommand{\R}{\mathbb{R}}
	\newcommand{\real}{\R}
	\newcommand{\C}{\mathbb{C}}
	\newcommand{\complex}{\C}
	\newcommand{\0}{\mathbb{O}}
	\newcommand{\F}{\mathbb{F}}
	\newcommand{\K}{\mathbb{K}}
    \newcommand{\angstrom}{\textup{\AA}}
    
%..PFEILE
	\renewcommand{\leadsto}{\Longrightarrow}
	\newcommand{\leftrightleadsto}{\Longleftrightarrow}

%..SCHRIFT
	\newcommand{\mbf}[1] {\mathbf{#1}}
	\newcommand{\mrm}[1] {\mathrm{#1}}
  \renewcommand{\phi}{\varphi}

%..VEKTOREN
	\newcommand{\Ul} {\underline}
	\newcommand{\vEx} {\vec{e}_x}
	\newcommand{\vEy} {\vec{e}_y}
	\newcommand{\vEz} {\vec{e}_z}
	\newcommand{\vEq} {\vec{e_1}}
	\newcommand{\vEw} {\vec{e_2}}
	\newcommand{\vEe} {\vec{e_3}}
	\newcommand{\transpose} {^{\text{T}}}
	\newcommand{\vect}[1]{\boldsymbol{#1}}
	
%..MATRIX
    \newcommand{\MATR}[1]{ \displaystyle \left[ \begin{matrix} #1 \end{matrix} \right]}
    \newcommand{\MATRABS}[1]{ \displaystyle \left| \begin{matrix} #1 \end{matrix} \right|}


%..KOMPLEXE ZAHLEN
	\renewcommand{\Re}{\text{Re}\,}
	\renewcommand{\Im}{\text{Im}\,}

%..OPERATOREN
	\DeclareMathOperator{\grad}{grad}
	\renewcommand{\div}{\text{div}\,}
    	\DeclareMathOperator{\rot}{rot}
    	\DeclareMathOperator{\divg}{div}
    	\DeclareMathOperator{\Tr}{Tr}
    	\DeclareMathOperator{\const}{const}
	\DeclareMathOperator{\imag}{i}
	\newcommand{\Lapl}{\hbox{\footnotesize{$\Delta$}}}

%..DIFFERENTIALRECHNUNG
	\newcommand{\Dx} {\,\mathrm{d}}
	\newcommand{\abl}[1] {\frac{\mathrm{d}}{\mathrm{d}#1}}
	\newcommand{\Abl}[2] {\frac{\mathrm{d}#1}{\mathrm{d}#2}}
	\newcommand{\ablq}[1] {\frac{\mathrm{d^2}}{\mathrm{d}#1^2}}
	\newcommand{\Ablq}[2] {\frac{\mathrm{d^2}#1}{\mathrm{d}#2^2}}
	\newcommand{\pabl}[1] {\frac{\partial}{\partial#1}}
	\newcommand{\pablq}[1] {\frac{\partial^2}{\partial#1^2}}
	\newcommand{\Pabl}[2] {\frac{\partial#1}{\partial#2}}
	\newcommand{\Pablq}[2] {\frac{\partial^2#1}{\partial#2^2}}

%..INTEGRALRECHNUNG
	\newcommand{\dint}{\displaystyle{\int}}
	\newcommand{\intab}{\int^b_a}
	\newcommand{\intinf}{\int_{-\infty}^\infty}
  \newcommand{\Int}{\int\displaylimits}
	\newcommand{\dintab}{\displaystyle{\int^b_a}}
	\newcommand{\dintpi}{\displaystyle{\int^{\pi}_{-\pi}}}
	\newcommand{\dintzpi}{\displaystyle{\int^{2\pi}_{\mbox{-}2\pi}}}
	\newcommand{\dA}{\hspace{4pt}\mathrm{d}A}
	\newcommand{\dx}{\hspace{4pt}\mathrm{d}x}
	\newcommand{\dy}{\hspace{4pt}\mathrm{d}y}
	\newcommand{\dz}{\hspace{4pt}\mathrm{d}z}
	\newcommand{\dr}{\hspace{4pt}\mathrm{d}r}
	\newcommand{\ds}{\hspace{4pt}\mathrm{d}s}
	\newcommand{\dS}{\hspace{4pt}\mathrm{d}S}
	\newcommand{\dt}{\hspace{4pt}\mathrm{d}t}
	\newcommand{\dm}{\hspace{4pt}\mathrm{d}m}
	\newcommand{\dk}{\hspace{4pt}\mathrm{d}k}
	\newcommand{\dl}{\hspace{4pt}\mathrm{d}l}
	\newcommand{\du}{\hspace{4pt}\mathrm{d}u}
	\newcommand{\dv}{\hspace{4pt}\mathrm{d}v}
	\newcommand{\dV}{\hspace{4pt}\mathrm{d}V}
	\newcommand{\dphi}{\hspace{4pt}\mathrm{d}\varphi}
	\newcommand{\domega}{\hspace{4pt}\mathrm{d}\omega}
	\newcommand{\dvarsigma}{\hspace{4pt}\mathrm{d}\varsigma}
	\newcommand{\dtau}{\hspace{4pt}\mathrm{d}\tau}
	\newcommand{\dtheta}{\hspace{4pt}\mathrm{d}\vartheta}
	\newcommand{\dmu}{\hspace{4pt}\mathrm{d}\mu}
	\newcommand{\dxi}{\hspace{4pt}\mathrm{d}\xi}
	\newcommand{\deta}{\hspace{4pt}\mathrm{d}\eta}
	\newcommand{\dvecl}{\hspace{4pt}\mathrm{d}\vec{l}}
	\newcommand{\dvecS}{\hspace{4pt}\mathrm{d}\vec{S}}

%..LIMES
    \DeclareMathOperator{\limni}{\lim\limits_{n\to\infty}}
    \DeclareMathOperator{\limxi}{\lim\limits_{x\to\infty}}
    \DeclareMathOperator{\limho}{\lim\limits_{h\to0}}
    \newcommand{\limxai}[1]{\ensuremath{\lim\limits_{x\to #1}}}

%..SUMMEN
    \DeclareMathOperator{\sumni}{\sum_{n=0}^{\infty}}
    \newcommand{\sumnia}[1]{\ensuremath{\sum_{n=#1}^{\infty}}}


%..PARTIELLE ABLEITUNG
    \DeclareMathOperator{\partf}{\dfrac{\partial f}{\partial x}}
    \newcommand{\partfo}[1]{\ensuremath{\dfrac{\partial f}{\partial #1}}}
    \newcommand{\parto}[1]{\ensuremath{\dfrac{\partial }{\partial #1}}}
    \newcommand{\partt}[2]{\ensuremath{\dfrac{\partial^2 }{\partial #1\partial #2}}}
    \newcommand{\partq}[1]{\ensuremath{\dfrac{\partial^2 }{\partial #1^2}}}


%..ENUMERATION
    \newenvironment{abc}{\begin{enumerate}[(a)]}{\end{enumerate}}
    \newenvironment{cabc}{\begin{compactenum}[(a)]}{\end{compactenum}}
    \newenvironment{romanenum}{\begin{enumerate}[i.]}{\end{enumerate}}
    \newenvironment{cromanenum}{\begin{compactenum}[i.]}{\end{compactenum}}

%..FUNCTIONS
    \DeclareMathOperator{\arsinh}{arsinh}
    \DeclareMathOperator{\arcosh}{arcosh}
    \DeclareMathOperator{\artanh}{artanh}
    \DeclareMathOperator{\arcoth}{arcoth}
    \DeclareMathOperator{\arccot}{arccot}
    \DeclareMathOperator{\Arg}{Arg}
    \DeclareMathOperator{\Log}{Log}
    \newcommand{\dis}[1]{\hspace{#1cm}}
    \newcommand{\abs}[1]{\ensuremath{\left\vert#1\right\vert}}
    \newcommand{\attention}{\raisebox{-1pt}{{\makebox[1.6em][c]{\makebox[0pt][c]{\raisebox{.13em}{\small!}}\makebox[0pt][c]{\color{red}\Large$\bigtriangleup$}}}}}
    \DeclareMathOperator{\meq}{\stackrel{!}{=}}
    
%..GRAPHICS
	\newcommand{\cgraphic}[2]{\begin{center}\includegraphics[width=#1\columnwidth,keepaspectratio]{#2}\end{center}}
	\newcommand{\mgraphic}[2]{\begin{wrapfigure}{l}{#1\linewidth}\includegraphics[width=\linewidth]{#2}\end{wrapfigure}}
    
% section color box
\setkomafont{section}{\mysection}
\newcommand{\mysection}[1]{%
    \Large\sffamily\bfseries%
    \setlength{\fboxsep}{0cm}%already boxed
    \colorbox{\StyleColor!40}{%
        \begin{minipage}{\linewidth}%
            \vspace*{2pt}%Space before
            #1
            \vspace*{-1pt}%Space after
        \end{minipage}%
    }}

%subsection color box
\setkomafont{subsection}{\mysubsection}
\newcommand{\mysubsection}[1]{%
    \normalsize \sffamily\bfseries%
    \setlength{\fboxsep}{0cm}%already boxed
    \colorbox{\StyleColor!20}{%
        \begin{minipage}{\linewidth}%
            \vspace*{2pt}%Space before
             #1
            \vspace*{-1pt}%Space after
        \end{minipage}%
    }}

% highlighter
\newcommand{\hilight}[1]{\colorbox{\StyleColor}{#1}}
\newcommand{\highlighty}[1]{%
  \setlength{\fboxsep}{0pt}\colorbox{yellow!100}{\ensuremath{#1}}}

\newcommand{\highlightg}[1]{%
  \setlength{\fboxsep}{0pt}\colorbox{green!100}{\ensuremath{#1}}}

\newcommand{\highlightbg}[1]{%
   \colorbox{green!100}{$\displaystyle #1$}}  

% equation box        
\newcommand{\eqbox}[1]{\setlength{\fboxrule}{1mm}\fcolorbox{\StyleColor}{white}{\hspace{0.5em}$\displaystyle#1$\hspace{0.5em}}}

%center equationbox
\newcommand{\ceqbox}[1]{\vspace*{4pt} \begin{center}\eqbox{#1}\end{center}\vspace*{4pt}}

%change page style for header
\pagestyle{fancy}
\footskip 20pt
\rhead{Marco Dober}
\lhead{Communication Networks}
\chead{\thepage}
\cfoot{}
\headheight 17pt \headsep 10pt
\title{Communication Networks}
\author{Marco Dober}
\date{\today}


\begin{document}
	\setcounter{secnumdepth}{3} %no enumeration of sections
	\begin{multicols*}{4}
		%
		\section*{Disclaimer}
			This should be a summary of the Communication Networks course. The goal is to update it weekly with the currently taught material.  	
			\newpage

		\maketitle 
		\thispagestyle{fancy}
		
		\section{Overview}
			\subsection{What is a network made of?}
				Networks are composed of three basic components:
				\begin{itemize}
					\item \textbf{End-systems} $\vert$ send \& receive data $\vert$  PC, Server, Smartphone, car navigation
					\item \textbf{Switches/Routers} $\vert$ forward data to destination $\vert$ vary in size and usage (home to data center)
					\item \textbf{Links} $\vert$ connect end-systems to switches and switches to each other $\vert$ copper, wireless, optical-fiber
				\end{itemize}
				\includegraphics[width= \columnwidth]{images/Overview/network_components.png}
				The internet is a network of networks. The Internet Service Providers (ISP) provide internet to their customers.\\
				\includegraphics[width= \columnwidth]{images/Overview/ISP.png}
				\columnbreak
			
				There exists a huge amount of \textbf{access technologies: }
				\begin{itemize}[noitemsep]
					\item \textbf{Ethernet} $\vert$ most common, symmetric (Up- and Down-stream same bandwidth)
					\item \textbf{DSL} $\vert$ phone lines, asymmetric (Up- and Down-stream NOT same bandwidth)
					\item \textbf{CATV} $\vert$ via cable TV, shared
					\item \textbf{Cellular} $\vert$ Smart phones 
					\item \textbf{Satellite} $\vert$ remote areas
					\item \textbf{FTTH} $\vert$ fiber to the home
					\item \textbf{Fibers} $\vert$ Internet backbone 
					\item \textbf{Infiniband} $\vert$ High performance computing 
				\end{itemize}
			\subsection{How is it shared?}
				So far we discussed the "last mile" of the Internet.\\
				3 must-have \textbf{requirements} of a good network topology: 
				\begin{itemize}
					\item \textbf{Tolerate failure} $\vert$ several path between src and dst.
					\item \textbf{Sharing to be feasible (praktikabel) \& cost effective } $\vert$ not too much links 
					\item \textbf{Adequate per-node capacity} $\vert$ not to few links 
				\end{itemize}
				The Design of the Internet is a mix of full-mesh, chain and bus which is an optimization of the above requirements. This topology is called a \textbf{switched network}.
				\vspace{-0.5cm}
				%\vspace{-\topsep}
				\begin{itemize}[noitemsep,topsep=0pt]
					\item \textcolor{ForestGreen}{Advantages:}
					\begin{itemize}
						\item \textcolor{ForestGreen}{Sharing and per-node capacity can be adapted to fit the network needs.}
					\end{itemize} 
					\item \textcolor{Red}{Disadvantages:}
					\begin{itemize}
						\item \textcolor{Red}{Require smart devices to perform; forwarding, routing, resource allocation (Zuweisung) }
					\end{itemize} 
				\end{itemize} 
				In a switched network links and switches are shared between flows.
				\includegraphics[width= \columnwidth]{images/Overview/link_switch_sharing.png}
				\columnbreak
				
				There exist two approaches of sharing, both are examples of statistical multiplexing: 
				\begin{itemize}
					\item \textcolor{red}{\textbf{Reservation}}\\
						  principle: reserve needed bandwidth in advance\\
						  multiplexing: at the flow-level\\
						  implementation: \textbf{circuit-switching}  
					\item \textcolor{red}{\textbf{On-demand}}\\
						  principle: send data when you need\\
						  multiplexing: at the packet-level\\
						  implementation: \textbf{packet-switching}
				\end{itemize}
				\textbf{Circuit-Switching:}
				\vspace{-0.5cm}
				\begin{itemize}[noitemsep]
					\item Relies on the Resource Reservation Protocol.
					\item The efficiency depends on how utilized the circuit is once established. The circuit can be mostly idle or just be used for a small amount of time (bad).
					\item It doesn't route around trouble 
				\end{itemize}
				\includegraphics[width=\columnwidth]{images/Overview/circuit_switching.png}
				\includegraphics[width=\columnwidth]{images/Overview/circuit_switching_transfer.png}
				\begin{itemize}[noitemsep]
					\item \textcolor{ForestGreen}{Advantages:}
					\begin{itemize}
						\item \textcolor{ForestGreen}{Predictable performance} 
						\item \textcolor{ForestGreen}{Simple \& fast switching (once circuit established)}
					\end{itemize}
					\item \textcolor{red}{Disadvantages:}
					\begin{itemize}
						\item \textcolor{red}{Inefficient if traffic is bursty or short}
						\item \textcolor{red}{Complex circuit setup/teardown (adds delay to transfer)}
						\item \textcolor{red}{Requires new circuit upon failure}
					\end{itemize} 
				\end{itemize}
				\columnbreak
				
				\textbf{Packet-Switching:}
				\vspace{-0.2cm}
				\begin{itemize}[noitemsep]
					\item Data transfer is done using independent packets 
					\item Since packets are not coordinated, they can clash with each other \
					\item To absorb transient overload, packet switching relies on buffers 
					\item It routes around trouble on the fly
				\end{itemize}
				\includegraphics[width=\columnwidth]{images/Overview/packet_switching_buffer.png}
				\begin{itemize}[noitemsep]
					\item \textcolor{ForestGreen}{Advantages:}
					\begin{itemize}
						\item \textcolor{ForestGreen}{Efficient use of resources} 
						\item \textcolor{ForestGreen}{Simpler to implement}
						\item \textcolor{ForestGreen}{Route around trouble}
					\end{itemize}
					\item \textcolor{red}{Disadvantages:}
					\begin{itemize}
						\item \textcolor{red}{unpredictable performance}
						\item \textcolor{red}{Requires buffer management and congestion (Stau) Control}
					\end{itemize} 
				\end{itemize}
				Packet-switching beats circuit-switching with respect to \textbf{resiliency} (robustness) and \textbf{efficiency}. 
				\includegraphics[width=\columnwidth]{images/Overview/internet_loves_packets.png}
			\newpage
			
			\subsection{How is it organized?}
				The Internet has a hierarchical structure and consists of about 60'000 networks: 
				\begin{itemize}[noitemsep]
					\item \textbf{Tier-1} (international)
					\begin{itemize}
						\item have no provider 
						\item $\approx$12 networks
					\end{itemize}
					\item \textbf{Tier-2} (national)
					\begin{itemize}
						\item provide transit to Tier-3s
						\item have at least one provider 
						\item $\approx$1'000s networks 	
					\end{itemize}
					\item \textbf{Tier-3} (local)
					\begin{itemize}
						\item do not provide any transit
						\item have at least one provider
						\item 85-90\% 
					\end{itemize}
				\end{itemize}
				\includegraphics[width=\columnwidth]{images/Overview/hirarchy.png}
				Some networks have an incentive (Anreiz) to connect directly, to reduce their bill with their own provider (direct traffic flow between them). This is known as \textcolor{Blue}{\textbf{peering}}\par
				
				\textbf{IXPs} (Internet Exchange Points): provide Internet connection for Tier2 and other providers. Only have \textcolor{Blue}{\textbf{peering-connections}} . 
				\includegraphics[width=\columnwidth]{images/Overview/IXPs.png}
				\subsection{How does communication happen?}
				Use \textbf{protocols} to enable communication between processes in different networks. Protocols are like a conversation convention. There are thousands of different protocols. Subdivide in different \textbf{layers} to keep stuff simple (Modularity).
				\begin{center}
					\textbf{5 Layer Model}\\
					\Rotatebox{270}{
						\begin{tabular}{l l l l l}
							  & Layer & service provided & role & protocol \\
							\hline						
							L5& Application  & network access & exchanges \textbf{messages} btw. proc. & HTTP, SMTP, FTP, SIP, ...  \\ 
							\hline 
							L4& Transport & end-to-end delivery & transport \textbf{segments} btw. end-sys.&TCP, UDP, SCTP \\ 
							\hline 
							L3& Network  & global best-effort delivery& move \textbf{packets} around the network&IP \\ 
							\hline 
							L2& Link & local best effort delivery& move \textbf{frames} across a link& Ethernet, Wifi, DSL, LTE,... \\ 
							\hline 
							L1& Physical & physical transfer bits & move \textbf{bits} across medium & copper, fiber, coax, ...\\ 
						\end{tabular}
					}
				\end{center}
				\par 
				Each layer provides a service to the layer above by using the layer below. Physical is foundation and everything is then built on top. 
				%\includegraphics[width=\columnwidth]{images/Overview/clock.png}
				Each layer has a \textbf{unit of data} and is implemented with different protocols and technologies (HW/SW). We can see shift to more HW because of speed. 
				\includegraphics[width=\columnwidth]{images/Overview/hw_sw.png}
				Each layer takes message from above and encapsulates with its own \textbf{header} and/or \textbf{trailer}. 
				\includegraphics[width=\columnwidth]{images/Overview/header_adding.png}
				\begin{itemize}[noitemsep]
					\item \textbf{Switches} act as a \textbf{L2 gateway}
					\item \textbf{Routers} act as a \textbf{L3 gateway}
				\end{itemize}
			
				\subsection{How do we characterize the network?}
				We characterize the network with:
				\begin{itemize}[noitemsep]
					\item \textbf{Delay}	
					\item \textbf{Loss}
					\item \textbf{Throughput} 
				\end{itemize}
				\textbf{Delay}
				\vspace{-0.2cm}
				\begin{itemize}[noitemsep]
					\item[$\rightarrow$] transmission $\vert$ link property
					\item[$\rightarrow$] propagation $\vert$ link property
					\item[$\rightarrow$] processing $\vert$ traffic $\vert$ mostly tiny
					\item[$\rightarrow$] queuing $\vert$ traffic $\vert$ hardest to evaluate
					\begin{itemize}
						\item[$-$] arrival rate at the queue
						\item[$-$] transmission rate of outgoing link
						\item[$-$] traffic burstiness 
					\end{itemize}
					$\text{traffic intensity} = \dfrac{L\cdot a}{R}$\\
					$a = \text{average packet arrival rate [packet/sec]} $\\
					$R = \text{transmission rate of outgoing link [bit/sec]} $\\
					$L = \text{fixed packet length [bit]} $	
				\end{itemize}
			\textbf{Loss}\\
			If the buffer of a queue is full, it drops packets and hence the packets are lost.\par
			\textbf{Throughput}\\
			To compute throughput one has to consider the bottleneck link
			\includegraphics[width=\columnwidth]{images/Overview/traffic_intensity.png}
			As technology improves , throughput increases \& delays are getting lower, except for propagation $\rightarrow$ content delivery networks move content closer to you (e.g. akamai).
			\includegraphics[width=\columnwidth]{images/Overview/akamai.png}
			\newpage 
			
			\section{Concepts}
			\subsection{Routing}
			How do you guide \textbf{IP packet}s from a source to a destination? \\
			Like an envelope, packets have a \textcolor{LimeGreen}{\textbf{header}} and a \textcolor{Orange}{\textbf{payload}}.\\ 
			\begin{center}
			\includegraphics[height=0.4\columnwidth]{images/Concepts/IP_packet.png}
			\end{center}
			Routers forward IP packets \textbf{hop-by-hop}. Routing is mostly not symmetrical (to/back not the same). Routers locally look up their \textbf{forwarding table} to know where to send the packet. Forwarding decisions necessarily depend on the \textbf{destination}, but also can depend on others (source, input port).\\
			In addition to data-plane routers also have a control plane consisting of:
			\vspace{-0.2cm}
			\begin{itemize}[noitemsep]
				\item Routing
				\item Configuration
				\item Statistics
				\item ... 
			\end{itemize}  
			\textbf{Routing} is the control-plane process that \textbf{computes} and \textbf{populates} the forwarding tables.
			\begin{center}
				\includegraphics[height=0.4\columnwidth]{images/Concepts/control_data_plane.png}
			\end{center}
			\textbf{Forwarding vs. Routing}\\
			\vspace{0.1cm}	
			\includegraphics[width=\columnwidth]{images/Concepts/forwarding_vs_routing.png}
			A global forwarding state is valid if and only if:
			\begin{itemize}[noitemsep]
				\item No dead ends
				\item No loops
			\end{itemize} 
		
			\subsubsection{Verifying that a forwarding state is valid}
			It's easy to verify that a routing state is valid. \\
			Simple algorithm: 
			\vspace{-0.1cm}
			\begin{enumerate}[noitemsep]
				\item Mark all outgoing ports with an arrow
				\item Eliminate all link with no arrow 
				\item Sate is valid iff the remaining graph is a \textbf{spanning-tree}
			\end{enumerate} 
			See the following pictures for an example with a resulting spanning tree and one with no spanning tree, hence no valid forwarding state.
			\includegraphics[width=\columnwidth]{images/Concepts/check_s_t.png}
			 
			\subsubsection{How to compute forwarding states}\label{kap:How to compute forwarding}
			Producing valid routing state is harder $\rightarrow$ prevent dead ends (easy) \& loops (hard). Prevent loops is the hard part, this is where routing protocols differ. There are three ways to compute valid routing state: 
			\begin{enumerate}[noitemsep]
				\item Use tree-like topologies $\vert$ \textbf{Spanning-tree}
			 	\item Rely on a global network view $\vert$ \textbf{Link-state}
			 	\item Rely on distributed computation $\vert$ \textbf{Distance vector} 
		 	\end{enumerate}
			In the Internet we use 3., because it is not possible to make precise map of whole Internet.\\
			In Networks we use 2.\\
			Inside (part of) Networks we use 1. \par
			
			\textbf{1. Spanning Tree}\\
			The easiest way to avoid loops is to route traffic in a loop free topology (Sherlock). Simple algorithm: 
			%\vspace{-0.2cm}
			\begin{enumerate}[noitemsep]
				\item Take an arbitrary topology
				\item Build a spanning tree and ignore all other links
				\item Done!
			\end{enumerate}
			It works, because spanning trees only have one path between any two nodes. There are numerous tress for a topology and they vary in efficiency. \\
			Once we have an spanning tree, forwarding is easy $\rightarrow$ just \textbf{flood} the packets everywhere (see picture below). This is very \textbf{inefficient}. 
			\includegraphics[width=\columnwidth]{images/Concepts/flooding_1.png}
			Solution: Nodes can \textbf{learn} how to reach nodes by remembering where packets came from. Ethernet works just like that. Learning is \textbf{topology-dependent!}\\
			Routing by flooding on a spanning tree (in a nutshell):
			\begin{itemize}[noitemsep]
				\item Flood first packet to node you're trying to reach\\ $\rightarrow$ all switches learn where you are
				\item When destination answers, some switches learn where it is\\
				$\rightarrow$ some because packet to you is not flooded anymore
				\item The decision to flood or not is done on each switch\\ $\rightarrow$ depending on who has communicated before 
			\end{itemize}
			\begin{itemize}[noitemsep]
				\item \textcolor{ForestGreen}{Advantages:}
				\begin{itemize}
					\item \textcolor{ForestGreen}{Plug-and-Play (no config. needed)} 
					\item \textcolor{ForestGreen}{Automatically adapts to moving host}
				\end{itemize}
				\item \textcolor{red}{Disadvantages:}
				\begin{itemize}
					\item \textcolor{red}{Mandate a spanning tree (eliminate many links form the topology)}
					\item \textcolor{red}{Slow to react to failures / host movements}
				\end{itemize} 
			\end{itemize}
			
			\textbf{2. Link-state}\\
			If \textbf{each routers} knows the entire graph, it can \textbf{locally} compute paths to all other nodes. Once a node \textit{u} knows the entire topology, it can compute shortest paths using \textbf{Dijkstra's algorithm}:
			\includegraphics[width=\columnwidth]{images/Concepts/Dijkstra.png}
			$u$ is the node running the algorithm\\
			$c(u,v)$ is the weight of the link connecting $u$ and $v$. \\
			$D(v)$ is the smallest distance currently known by $u$ to reach $v$.\\
			\includegraphics[width=\columnwidth]{images/Concepts/Dijkstra_example.png} 
			Normally the algorithm has $\mathcal{O}(n^2)$ complexity ($n$ being the number of nodes), but with the help of a heap (data-structure...) the complexity can be brought down to $\mathcal{O}(n\log n)$, which is really \textbf{efficient}.\\
			From the shortest paths, $u$ can directly compute its forwarding table! 
		    \includegraphics[width=\columnwidth]{images/Concepts/forwarding_table.png} 	 
		    \textbf{How do we know cost?}\\
		   	Initally, routers only knwo their ID and their neighbors and the cost to reach them. They then build message known as Link-state (with neighbors and their cost/weight) and flood it in the network\\
		    $\rightarrow$ At the end of the flooding process everyone should have the exact same view of the network. \\
		    I can configure wight of links \textbf{static} by hand (Dijkstra will converge), or \textbf{dynamic} (may be problematic).\par 
		    
		    \textbf{3.Distance-vector}\\
		    Paths can be computed in distributed computation.\\
		    Let $d_x(y)$ be the cost of the least-cost path known by $x$ to reach $y$. \\
		    Each node bundles these distances into one message (vector) that it \textbf{repeatedly} sends to all its neighbors. \\
		    Each node updates its distances based on neighbors vectors:\\
		    $d_x(y)=\min\{c(x,v)+d_v(y)\}$, overall neighbors $v$.\\
		    This leads to a recursive computation of the shortest path, the result must be the same as Dijkstra algorithm! 
		    Example: Compute shortest path from $u$ to D:\\
		    $d_u(D)=\min\{c(u,A)+d_A(D), c(u,E)+d_E(D)\}$\\
		    $\downarrow$\\
		    $d_A(D)=\min\{c(A,B)+d_B(D), c(A,C)+d_C(D)\}=\min\{2+1,1+4\}=3$\\
		    $\downarrow$\\
		    $d_E(D)=\min\{c(E,C)+d_C(D), c(E,G)+d_G(D),c(E,u)+d_u(D)\}=5$\\
		    $\downarrow$\\
		    $d_u(D)=\min\{3+d_A(D), 2+d_E(D)\}=6$\\
		    As before $u$ can directyl infer its forwarding table, by directing the traffic to its best neighbor (the one which advertises the smallest cost). Evaluating the complexity of DV is harder. 
		    
		    \subsection{Reliable delivery}\label{Subsec:Reliable_Delivery}
		    How do you ensure reliable transport on top of best-effort delivery?\\
		    \textbf{Goals:}
		    \begin{itemize}[noitemsep]
		    	\item Keep the network simple,dumb\\
		    	$\rightarrow$ make it easy to build/operate network
		    	\item Keep application as network unaware as possible\\
		    	$\rightarrow$ Developer should focus on app, not network 
		    \end{itemize}
	    	\textbf{Design:}
	    	\begin{itemize}
	    		\item Implement reliability in-between, in the networking stack\\
	    		$\rightarrow$ relive the burden from both the app and network
	    	\end{itemize}
    		The Internet puts \textbf{reliability in L4}, just above the network layer.\\
    		What can the mean Internet do to our IP-packets:
    		\begin{itemize}[noitemsep]
    			\item Lost or delayed
    			\item Corruption
    			\item Reordering
    			\item Duplication 
    		\end{itemize}
    		We have four goals of reliable transfer:
    		\begin{itemize}[noitemsep]
    			\item \textbf{Correctness:} ensure data is delivered, in order, and untouched
    			\item \textbf{Timeliness:} minimize time until data is transferred 
    			\item \textbf{Efficiency:} optimal use of bandwidth
    			\item \textbf{Fairness:} play well with concurrent communications  
    		\end{itemize}
    		\textbf{Correctness:}\\
    		A reliable transport design is correct iff:\\
    		A packet is \textcolor{Red}{always resent} if the previous packet was lost or corrupted. A packet may be resent at other times.\\
    		Note: It is \textcolor{Red}{ok to give} up after a while, but it must be announced to the application.\par
    		
    		Designing a \textcolor{Red}{correct}, \textcolor{Red}{timely}, \textcolor{Red}{efficient} transport mechanism knowing that packets can get \textcolor{Red}{lost} (focus on mentioned aspects):
    		\includegraphics[width=\columnwidth]{images/Concepts/send_packet.png} 	 
    		There is a clear tradeoff between timeliness and efficiency in the selection of the timeout value. Big challenge to choose optimal value.\\
    		Small timers: Risk of \textbf{unnecessary retransmissions}\\
    		Large timers: Risk of \textbf{slow transmission} \par
    		
    		To improve timeliness just send multiple packets at the same time and not wait to ACK every packet. \\
    		Approach: 
    		\begin{itemize}[noitemsep]
    			\item Add sequence number to every packet
    			\item Add buffers to sender and receiver: 
    			\begin{itemize}
    				\item[$-$] Sender: store packets sent \& not acknowledged
    				\item[$-$] Receiver: store out-of-order packets received
    			\end{itemize}
    		\end{itemize} 
    		\includegraphics[width=\columnwidth]{images/Concepts/packets_wo_ack.png} 	 
    		One problem that can occur is: 
    		\includegraphics[width=\columnwidth]{images/Concepts/packets_wo_ack_problem.png} 
    		To solve this issue  we need a mechanism for \textcolor{Red}{\textbf{flow control}}.
    		Using a \textbf{sliding window} is one way to do that:
    		\begin{itemize}
    			\item[$-$] Sender keeps a list of the sequence \# it can send.\\
	   			$\rightarrow$ known as the \textit{sending window}
	   			\columnbreak
	   			\item[$-$] Receiver keeps a list of acceptable sequence \#\\
	   			$\rightarrow$ known as the \textit{receiving window }
	   			\item[$-$] Sender and receiver negotiate the window size\\
	   			$\rightarrow$ sending window $\le$ receiving window
    		\end{itemize} 
    		Example with a window-size of 4 packets:
    		\includegraphics[width=\columnwidth]{images/Concepts/windowsize4_1.png} 
    		%\vspace{0.2cm}
    		Window after sender receives \textcolor{Red}{ACK 4 }
    		%\vspace{0.1cm}
    		\includegraphics[width=\columnwidth]{images/Concepts/windowsize4_2.png} 
    		Timeliness of the window protocol depends on the size of the sending window.\par
    		
    		\textbf{ACKing individual packets}
    		\vspace{-0.2cm}
    		\begin{itemize}[noitemsep]
    			\item \textcolor{ForestGreen}{Advantages:}
    			\begin{itemize}
    				\item \textcolor{ForestGreen}{Know fate of each packet} 
    				\item \textcolor{ForestGreen}{Simple window algorithm}
    				\item \textcolor{ForestGreen}{Not sensitive to reordering}
    			\end{itemize}
    			\item \textcolor{red}{Disadvantages:}
    			\begin{itemize}
    				\item \textcolor{red}{Unnecessary retransmission upon losses}
    			\end{itemize} 
			\end{itemize}
		
    		\textbf{Cumulative ACKs}\\
   			ACK the highest sequence number for which all the previous packets have been received.
   			\vspace{-0.2cm}
   			\begin{itemize}[noitemsep]
   				\item \textcolor{ForestGreen}{Advantages:}
   				\begin{itemize}
   					\item \textcolor{ForestGreen}{Recover from lost ACKs}
   				\end{itemize}
   				\item \textcolor{red}{Disadvantages:}
   				\begin{itemize}
   					\item \textcolor{red}{Causes unnecessary retransmissions}
   					\item \textcolor{red}{Confused by reordering}
   					\item \textcolor{red}{Incomplete information about which packets have arrived}
   				\end{itemize} 
   			\end{itemize}
   		
   			\textbf{Full information feedback}\\
   			List all packets that have been received highest cumulative ACK, plus any additional packets
   			\vspace{-0.2cm}
   			\begin{itemize}[noitemsep]
   				\item \textcolor{ForestGreen}{Advantages:}
   				\begin{itemize}
   					\item \textcolor{ForestGreen}{Complete information}
   					\item \textcolor{ForestGreen}{Resilient (belastbar) form of individual ACKs}
   				\end{itemize}
   				\item \textcolor{red}{Disadvantages:}
   				\begin{itemize}
   					\item \textcolor{red}{Overhead}
   				\end{itemize} 
   			\end{itemize}
   			
   			With individual ACKs and full information detection of a missing packet is easy (implicit/explicit). \\
   			With cumulative ACKs missing packets are harder to know. \par
   			
   			\textbf{Fairness:}\\
   			Fair is mostly not efficient. Defining what fair is, is not easy. What matters is to \textbf{avoid starvation}. Equal-per-flow is good enough for this. Simply dividing available bandwidth doesn't work.\\
   			We want to give users with small demands what they want and evenly distribute the rest. \textbf{max-min fair allocation} is such that: the lowest demand is maximized $\rightarrow$ the second lowest is maximized $\rightarrow$ the third lowest is maximized and so on...\\
   			\textbf{max-min fair allocation:}
   			\begin{enumerate}[noitemsep]
   				\item Start with all flows at rate 0
   				\item Increase the flows until there is a new bottleneck in the network
   				\item Hold the fixed rate of the flows that are bottlenecked
   				\item Got to step 2 for the remaining flows. 	 
   			\end{enumerate}  
   			Max-min fair allocation can be approximated by increasing window until a loss is detected. \par
   			
   			\textbf{Corruption:}\\
   			Dealing with corruption is easy: Rely on a checksum and treat corrupted packets as lost ones. \par 
   			
   			\textbf{Reordering:}\\
   			Effect depends on ACKing mechanism which is used:\\
   			$\bullet$ Individal ACK: \textcolor{ForestGreen}{no problem}\\
   			$\bullet$ Full feedback: \textcolor{ForestGreen}{no problem}\\
   			$\bullet$ Cumm. ACKs: \textcolor{Red}{Create duplicate ACKs}.\par
   			
   			\textbf{Duplicates:}\\
   			Can lead to duplicated ACKs whose effect depends on the ACKing mechanism: \\
   			$\bullet$ Individual ACK: \textcolor{ForestGreen}{no problem}\\
   			$\bullet$ Full feedback: \textcolor{ForestGreen}{no problem}\\
   			$\bullet$ Cumm. ACKs: \textcolor{Red}{problematic}\par
   			
   			\textbf{Delay:}\\
   			Can create useless timeouts for all designs. It is hard to deal with the different delays which occur over the whole network. How do I set the right amount of timeout?\\
   			\newpage
   			
   			\section{The Link Layer}
   			\subsection{What is a link?}
   			\textbf{Link =  Medium + Adapter}\\
   			Network adapters communicate together through the medium.\\
   			In the link layer we talk about \textbf{Frames} which are sent form one adapter to the other.\\
   			\includegraphics[width=\columnwidth]{images/Link_Layer/link.png} 
   			\textbf{Sender}
   			\vspace{-0.3cm}
   			\begin{itemize}[noitemsep]
   				\item Encapsulate packets in a frame
   				\item add error checking bits, flow control,...
   			\end{itemize}
   			\textbf{Receiver:}
   			\vspace{-0.3cm}
   			\begin{itemize}[noitemsep]
   				\item Look for errors, flow control,...
   				\item Extract packet and passes it to network layer 
   			\end{itemize}
   			Link-Layer provides a best effort delivery service to the network layer, composed od \textbf{5} sub services.
   			\begin{itemize}[noitemsep]
   				\item \textbf{Encoding} $\vert$ represents the 0's and 1's
   				\item \textbf{Framing} $\vert$ encapsulates packet into a frame (header/trailer)
   				\item \textbf{Error detection} $\vert$ detects errors with checksum
   				\item \textbf{error correction} $\vert$ optionally correct errors
   				\item \textbf{Flow Control} $\vert$ pace sending and receiving node 
   			\end{itemize} 
   			
   			\subsection{How do we identify link adapters?}
   			\textbf{M}edium \textbf{A}ccess \textbf{C}ontrol address:
   			\vspace{-0.2cm}
   			\begin{itemize}[noitemsep]
   				\item \textbf{Identify the sender \& receiver adapters} $\vert$ used within a link
   				\item \textbf{Are uniquely assigned} $\vert$ hard coded into adapter
   				\item \textbf{Use a flat space of 48 bits} $\vert$ allocated hierarchically 
   			\end{itemize}
   			The \textcolor{Green}{first} 24 bits blocks are assigned to network adapter vendor by IEEE. (1 Vendor may have more than 1 block.)
   			\includegraphics[width=\columnwidth]{images/Link_Layer/first_mac_block.png}
   			The \textcolor{Orange}{second} 24 bits block is assigned by the vendor to each network adapter.(My use the same in geographically different locations!)
   			\includegraphics[width=\columnwidth]{images/Link_Layer/second_mac_block.png}
   			broadcast address has set all bits to 1: ff:ff:ff:ff:ff:ff, enables to send a frame to all adapters on the link.\\
   			By default, adapters only decapsulate frames addressed to the local MAC or broadcast address. The promiscuous mode enables to decapsulate everything.\\
   			Why don't we simply use IP-addresses?
   			\begin{enumerate}[noitemsep]
   				\item Links can support any protocol (not just IP) $\vert$ different addresses on different kind of links
   				\item Adapters may move to different locations $\vert$ cannot assign static IP address, it has to change
   				\item \textbf{Adapters must be identified during bootstrap} $\vert$ need to talk to an adapter to give it an IP address 
   			\end{enumerate} 
   			You need to solve two problems when bootstrapping an adapter:
   			\begin{itemize}[noitemsep]
   				\item Who am I? (How do I acquire an IP address $\vert$ MAC-to-IP binding) $\vert$ \textcolor{Red}{Dynamic Host Configuration Protocol DHCP}
   				\item Who are you? (Given an reachable IP-address on a link, how do I find out what MAC to use) $\vert$ IP-to-MAC binding $\vert$ \textcolor{Red}{Address Resolution Protocol ARP}
   			\end{itemize}
   			Network adapters traditionally acquire an IP address using \textbf{DHC}:
   			\begin{enumerate}[noitemsep]
   				\item \textbf{Discovery} $\vert$ Client searching DHCP Server (via broadcast)
   				\item \textbf{Offer} $\vert$ DHCP server sending offer to client
   				\item \textbf{Request} $\vert$ Client request IP from DHCP server
   				\item \textbf{ACK} $\vert$ DHCP server assigns IP and sends ACK 
   			\end{enumerate}
   			\includegraphics[width=\columnwidth]{images/Link_Layer/DHCP.png}
   			\par 
   			
   			The \textbf{ARP} enables to discover the MAC associated to an IP: 
   			\begin{enumerate}[noitemsep]
   				\item ARP \textbf{Request}: Who has $<$some Ip$>$ tell $<$my IP$>$ to broadcast MAC
   				\item ARP \textbf{Reply}: $<$some IP$>$ is at $<$this MAC$>$
   				\item Requester puts entry in his \textbf{ARP-table}
   			\end{enumerate}
   			\includegraphics[width=\columnwidth]{images/Link_Layer/ARP.png}
   			
   			\subsection{How do we share a network medium?}
   			Some medium are \textbf{multi-access}: $>1$ host can communicate at same time.\\
   			\textcolor{Red}{Problem: Collision lead to garbled (verstümmelt) data.}\\
   			\textcolor{ForestGreen}{Solution: Distributed algorithm for sharing the channel.}\\
   			Essentially there are three techniques to deal with Multi Access Control (MAC):
   			\begin{itemize}[noitemsep]
   				\item \textbf{Divide the channel into pieces} $\vert$ either in time or frequency
   				\item \textbf{Take turns} $\vert$ pass a token for the right to transmit
   				\item \textbf{Random access} $\vert$ allow collisions, detect them and then recover
   			\end{itemize} 
   			
   			\subsection{What is Ethernet?}
   			\begin{itemize}[noitemsep]
   				\item Was invented as a broadcast technology
   				\item Is the dominant wired LAN technology 
   				\item Has managed to keep up with the speed race
   			\end{itemize}
   			Ethernet offers an \textbf{unreliable} and \textbf{connectionless} service.\\
   			Unreliable:
   			\begin{itemize}[noitemsep]
   				\item[$-$] Receiving adapter does not acknowledge anything
   				\item[$-$] Packets passed to the networks layer can have gaps
   			\end{itemize}
   			Connectionless: 
   			\begin{itemize}[noitemsep]
   				\item[$-$] No handshake between sender and receiver
   			\end{itemize}
   			Traditional Ethernet relies on CSMA/CD (carries-sense multiple access with collision detection). All hosts were on a big bus-cable connected. You needed to sense the cable in order to know if someone is speaking. multiple hosts had access to the medium, while you were speaking you still were listening to detect collisions. CSMA/CD imposes \textbf{limits} to the network length.\\
   			$\mathrm{Network\ length}=\frac{min\ frame\ size\ \cdot\ speed\ of\ light}{2\cdot bandwidth}$\\
   			For this reason Ethernet imposes a minimum packet size of 512 bits.\\
   			Modern Ethernet links interconnect exactly two hosts, in full duplex, rendering collisions impossible. 
   			\begin{itemize}[noitemsep]
   				\item CSMA/CD is only needed for half-duplex communication
   				\item This means the 64 Byte restriction is not strictly needed (but still kept)
   				\item Multiple Access Protocols are still important for wireless communication
   			\end{itemize}
   			The Ethernet header is simple, composed of 6 fields: 
   			\includegraphics[width=\columnwidth]{images/Link_Layer/eth_header.png}
			Ethernet efficiency is $\approx97.5\%$ (paylod/framesize). 
			
			\subsection{How do we interconnect segments at the link layer?}
			Historically, people connected Ethernet segments together at the physical level with ethernet \textbf{hubs}. Hubs work by repeating bits from one port to all the others (flooding everything). 
			\includegraphics[width=\columnwidth]{images/Link_Layer/hub.png}
			\begin{itemize}[noitemsep]
				\item \textcolor{ForestGreen}{Advantages:}
				\begin{itemize}
					\item \textcolor{ForestGreen}{Cheap, simple}
				\end{itemize}
				\item \textcolor{red}{Disadvantages:}
				\begin{itemize}
					\item \textcolor{red}{Inefficient}
					\item \textcolor{red}{Limited ot one LAN technology}
					\item \textcolor{red}{Limited number of nodes/distances} 
				\end{itemize} 
			\end{itemize}
   			LANs are now almost exclusively composed of Ethernet \textbf{switches}. Switches connect two or more LANs together on the Link layer, actings as L2 gateways. Switches are \textbf{store and forward} devices, they: 
   			\begin{itemize}[noitemsep]
   				\item Extract the DST MAC $\vert$ from the frame
   				\item Look up the MAC in a table $\vert$ using exact match
   				\item Forward the frame $\vert$ on the appropriate port
   			\end{itemize}
   			Similiar to IP routers, except they are one layer below. Switch enables each LAN segment to carry its own traffic (no flooding of network). 
   			\includegraphics[width=\columnwidth]{images/Link_Layer/switch_network.png}
   			Switches are Plug-and-Play they build their forwarding table on their own.\\
   			The advantages of switches are numerous: 
   			\begin{itemize}[noitemsep]
   				\item \textcolor{ForestGreen}{Advantages:}
   				\begin{itemize}
   					\item \textcolor{ForestGreen}{Only forward frames where needed} $\vert$ avoids unnecessary load on segments
   					\item \textcolor{ForestGreen}{Join segments using different technologies}
   					\item \textcolor{ForestGreen}{Improved privacy} $\vert$ hosts can only snoop traffic traversing their segment
   					\item \textcolor{ForestGreen}{Wider geographic span} $\vert$ separates segments allow longer distance
   				\end{itemize} 
   			\end{itemize}
   			\newpage
   			
   			When Frames arrive:
   			\vspace{-0.2cm} 
   			\begin{itemize}[noitemsep]
   				\item [$-$] Inspect the SRC MAC address
   				\item [$-$] associate the address with the port
   				\item [$-$] Store the mapping in the switch table
   				\item [$-$] Launch a timer to eventually forget the mapping
   			\end{itemize}
   		In case of misses, switches simply flood the network (when in doubt, shout).\\
   		When a frame arrives with an unknown DST $\rightarrow$ forward the frame out all ports, except where the frame arrived from.\\ 
   		Flooding enables \textbf{automatic discovery} of hosts, but with loops the load increases exponentially! Solution: Reduce the network to one logical \textbf{spanning tree}.\\
   		In practice, switches run distributed Spanning-Tree-Protocol (\textbf{STP}). \\
   		Construction of a spanning tree in a nutshell: Switches...
   		\begin{itemize}[noitemsep]
   			\item Elect a root $\vert$ the one with the smallest identifier
   			\item determine if each interface is on the shortest path from the root $\vert$ if not $\rightarrow$ disable it.
   		\end{itemize}
   		For this switches exchange Bridge Protocol Data Unit (BPDU) messages\\
   		Each Switch X iteratively sends: 
   		\includegraphics[width=\columnwidth]{images/Link_Layer/BPDU.png}
   		Initially: 
   		\vspace{-0.2cm}
   		\begin{itemize}[noitemsep]
   			\item Each switch proposes itself as root $\vert$ sends (X,0,X) on all its interfaces
   			\item Upon receiving (Y,d,X), checks if Y is a better root $\vert$ if so, consider Y as new root, flood updated message 
   			\item Switch compute their distance to the root, for each port $\vert$ simply add 1 to the distance received, if shorter, flood
   			\item Switches disable interfaces not on shortest-path
   		\end{itemize}
   		tie-breaking: 
   		\vspace{-0.2cm}
   		\begin{itemize}[noitemsep]
   			\item Upon receiving $\ne$ BPDUs from $\ne$ switches with $=$ cost $\rightarrow$ pick the BPDU with the lower siwtch sender ID
   			\item Upon receiving $\ne$ BPDUs from a neighboring switch $\rightarrow$ Pick the BPDU with the lowest port ID. 
   		\end{itemize}
   		To be robust, STP must react to failures: 
   		\vspace{-0.2cm}
   		\begin{itemize}[noitemsep]
   			\item Any switch, link or port can fail $\vert$ including the root switch
   			\item Root switch continuously send messages $\vert$ announcing itself as the root (1,0,1), others forward it
   			\item Failures are detected through timeout (soft state) $\vert$ if no word from root in X, times out and claims to be root 
   		\end{itemize}
   		\subsection{Virtual Local Aerea Networks VLANs}
   		The Local Area Networks we have considered so far define single broadcast domains (if one broadcasts, everyone receives it).\\
   		As the networks scales, operators like to \textbf{segment} thir LANs.\\
   		Why?
   		\begin{itemize}[noitemsep]
   			\item Improves \textbf{security} $\vert$ smaller attack surface (visibility \& injection)
   			\item Improves \textbf{performance} $\vert$ limits the overhead fo broadcast traffic 
   			\item Improves \textbf{logistics} $\vert$ separate traffic by role (staff, student,...)
   		\end{itemize} 
   		You do not want to separate your LAN physically (huge pain), but reader do it in software $\rightarrow$ \textbf{Virtual} Networks.\\
   		Definition:\\
   		A VLAN identifies a set of ports attached to one or more Ethernet Switches, forming one broadcast domain. 
   		\includegraphics[width=\columnwidth]{images/Link_Layer/VLAN.png}
   		Switches need \textbf{configuration} tables telling them which VLANs are accessible via which interface.\\
   		To identify VLAN, switches add \textbf{new header} when forwarding traffic to another switch.  
   		\includegraphics[width=\columnwidth]{images/Link_Layer/VLAN_header.png}
   		With VLANs, Ethernet links are divided in two sets: \textcolor{Blue}{\textbf{access}} and \textcolor{Red}{\textbf{trunks}} (inter switch) links.\\
   		Trunks carry traffic for more than one VLAN (access not), there the new header is needed! On the access links not! Communication between VLANs goes over a router!
   		Each switch runs one MAC learning algorithm for each VLAN.
   		\begin{itemize}[noitemsep]
   			\item When a switch receives a frame wit an unknown or broadcast DST.
   			\begin{itemize}
   				\item [$\rightarrow$]	it forwards it on all the ports that belong to the same VLAN
   			\end{itemize}
   		\columnbreak
   			\item  When a switch learns a SRC address on a port
   			\begin{itemize}
   				\item [$\rightarrow$] it associates it to the VLAN of this port and only uses it when forwarding frames to this VLAN	
   			\end{itemize}
   		\end{itemize}
  		\includegraphics[width=\columnwidth]{images/Link_Layer/trunks_access.png}
   		Switches can also compute per VLAN spanning trees $\rightarrow$ distinct SPT for each VLAN! This enables better use of the network. 
   		
   		\section{The Network Layer}
   		\subsection{IP addresses}
   		IPv4 addresses are unique 32-bits number associated to a network interface (on a host, router). IP addresses are usually written using dotted-quad notation. 
   		\includegraphics[width=\columnwidth]{images/Network_Layer/IP_address.png}
   		Routers forward packets based on their DST IP address. If IP addresses were assigned arbitrarily routers would require forwarding table entries for all of them!\\
   		IP addresses are hierarchical, composed of a\textcolor{Red}{ prefix (network address)} and a \textcolor{ForestGreen}{suffix (host address)}. 
   		\includegraphics[width=\columnwidth]{images/Network_Layer/prefix_suffix.png}
   		Each prefix has a given length, usually written using the slash notation:\\
   		IP prefix: 82.130.102.0/\textcolor{Red}{24}$\rightarrow$ prefix length in bit\\
   		The suffix part can then be used to address the hosts of the network.\\
   		$\bullet$The \textbf{first address} of the suffix (all zeros) is used to identify the \textbf{network itself}.\\
   		$\bullet$ The \textbf{last address} of the suffix (all ones) is used ot determine the \textbf{broadcast} address.\\
   		\# of hosts $= 2^{32-\#\mathrm{\textcolor{Red}{suffix}}}-2$\\
   		Prefixes are also sometimes specified using an address and a \textbf{mask}. ANDing the address and the mask gives you the prefix. \\
   		Mask: set the number of suffix bits to one, the rest zeros (from left to right).\\
   		\includegraphics[width=\columnwidth]{images/Network_Layer/mask.png}
   		Routers forward IP packets based on the network part, not the host part. This enables a scaling of the forwarding table. \\
   		Originally there were only 5 fixed allocation sizes (classes) - known as classful networking. This is wasteful and leaded to IP address exhaustion.\\
   		\includegraphics[width=\columnwidth]{images/Network_Layer/classes.png}
   		Problem: Class C was too small, so everybody requested B\\
   		Solution: Classless Inter-Domain Routing (CIDR) (1993)\\
   		CIDR enables flexible division between network and host addresses. 
   		\begin{itemize}[noitemsep]
   			\item CIDR must specify both, the address and the mask $\vert$ classful was communicating this in the first address bits
   			\item Masks are carried by the routing algorithms $\vert$ it is not implicitly carried in the address.  
   		\end{itemize}
   		With CIDR the maximal waste is bounded to 50\%.\\
   		Today, addresses are allocated in contiguous chunks:\\ 
   		\includegraphics[width=\columnwidth]{images/Network_Layer/IP_chunk.png}
   		\pagebreak
   		
   		The allocation process of prefixes is also hierarchical:\\
   		\includegraphics[width=\columnwidth]{images/Network_Layer/prefix_allocation.png}
   		
   		\subsection{IP forwarding}
   		Whats inside an IP router?:\\
   		$\bullet$ Data Plane:\\
   		\includegraphics[width=\columnwidth]{images/Network_Layer/data_plane.png}
   		$\bullet$ Control Plane:\\
   		\includegraphics[width=\columnwidth]{images/Network_Layer/control_plane.png}
   		Routers maintain \textbf{forwarding entries} for each Internet prefix.\\
   		When a router receives an IP packet, it performs an IP \textbf{look up} to find the matching prefix.\\
   		CIDR makes forwarding harder, as one packet can match many IP prefixes.\\
   		\includegraphics[width=\columnwidth]{images/Network_Layer/two_matches.png}
   		To resolve ambiguity, forwarding is done along the \textbf{most specific} prefix (IF\#3)\\
   		A child prefix can be \textbf{filtered} from the table whenever it shares the same output interface as its parent:
   		\includegraphics[width=\columnwidth]{images/Network_Layer/filtering1.png}
   		\includegraphics[width=\columnwidth]{images/Network_Layer/filtering2.png}
   		\textcolor{Red}{Exactly the same forwarding!}
   		
   		\subsection{IPv4 packet}
   		This is what an IPv4 Packet looks like:
   		\includegraphics[width=\columnwidth]{images/Network_Layer/IPv4_packet.png}
   		\textbf{Version} $\vert$  "4" or "6", tells us what other fields to expect\\ 
   		\textbf{Header length} $\vert$ Denotes the number of 32-bits word in the header (typically 5 (20 Bytes)) \\ 
   		\textbf{ToS} $\vert$ Allows different pakctes to be treated differently (low delay (VoIP), high bandwidth (Video))\\
   		\textbf{Total length} $\vert$ Denotes the number of Bytes in the entire packet (max. 65 635, mostly 1500 because ether.)\\
   		\textbf{Identification} $\vert$ Uniquely identifies the fragments of a particular packet.\\
   		\textbf{Flag} $\vert$ Tells you if more packets are coming or not.\\
   		\textbf{Fragment Offset} $\vert$ Used if need to reorder packets in right order.\\
   		\textbf{Time To Live} $\vert$ Used to identify packets in a loop and drop them.\\
   		\textbf{Protocol} $\vert$ Identifies the higher level protocol carried in the packet ("6" for TCP, "17" for UDP).\\
   		\textbf{Header Checksum} $\vert$ Sum of all 16 bits words in the header (does not protect the payload).\\
   		\textbf{Source IP} $\vert$ Uniquely identifies the Source host.\\
   		\textbf{Destination IP} $\vert$ Uniquely identifies the Destination host.\\
   		\textbf{Options} $\vert$ Used to provide flexibility, but mostly deactivated because security. (record route, strict source route, loose source route, timestamp, traceroute, router alert)\\
   		
   		\subsection{IPv6 packet}
   		coming soon
   		
   		\subsection{Internet routing}
   		Internet routing comes in two flavors: \textcolor{Red}{\textbf{intra}}- and \textcolor{ForestGreen}{\textbf{inter}}-domain routing.
   		\includegraphics[width=\columnwidth]{images/Network_Layer/intra_vs_inter.png}
   		\textcolor{Red}{\textbf{Intra}} vs. \textcolor{ForestGreen}{\textbf{Inter}} domain routing
   		\begin{center}
   				\includegraphics[width=0.5\columnwidth]{images/Network_Layer/intra_vs_inter_2.png}
   		\end{center}
   	
   		\subsubsection{Intra-domain routing}
   		Find paths \textbf{within} a network.\\
   		Intra-domain routing enables routers to compute \textbf{good forwarding paths} to any internal subnet. What is \textbf{good}?\\
   		\textbf{Definition:} A good path is a path that minimizes some network wide metric (cost, delay load, loss,...).\\
   		\textbf{Approach:} Assign to each link wa weight (usually static), compute the \textit{shortest path} to each destination.\par
   		
   		\textbf{Link-state protocols}\\
   		 In link-state routing, routers build a precise map of the network by flooding local views to everyone. 
   		 \begin{itemize}[noitemsep]
   		 	\item Each router keeps track of its incident links and cost $\vert$ as well as weather it is up or down
   		 	\item Each router broadcast its own links state $\vert$ to give every router a complete view of the graph
   		 	\item Routers run Dijkstra on the corresponding graph $\vert$ to compute their shortest path and forwarding table. 
   		 \end{itemize}
   	 	Flooding is performed as in the L2 learning, except that it is \textbf{reliable}. All nodes are \textbf{ensured} to receive the latest version of all link-states.\\
   	 	\begin{itemize}[noitemsep]
   	 		\item Challanges: 
   	 		\begin{itemize}
   	 			\item[$-$] packet loss
   	 			\item[$-$] out of order arrival
   	 		\end{itemize}
    		\item Solution:
    		\begin{itemize}
    			\item[$-$] ACK \& Retransmission
    			\item[$-$] Sequence number
    			\item[$-$] Time-To-Live for each link-state
    		\end{itemize}
   	 	\end{itemize}  
   		A link-state node initates flooding in 3 conditions: 
   		\begin{itemize}[noitemsep]
   			\item \textbf{Topology change} $\vert$ link or node failure/recovery
   			\item \textbf{Configuration change} $\vert$ link cost change
   			\item \textbf{Periodically} $\vert$ Refresh (account for possible data corruption)
   		\end{itemize}
   		Once a node knows the entire topology, it can run Dijkstra to compute shortest-path.\\
   		By default, link-state protocols detect changes using software based beaconing:\\
   		Routers periodically exchange "Hello" in both directions and trigger a failure after few missed "Hellos". Tradeoff between: 
   		\begin{itemize}[noitemsep]
   			\item Detection speed
   			\item Bandwidth and CPU overhead
   			\item false positive/negative
   		\end{itemize}  
   		During network changes, the link-state database of each node might differ. Inconsistencies lead to transient disruptions in the form of black holes or loops.\\
   		\textbf{Blackholes} appear due to detection delay, as nodes do not immediately detect failure $\rightarrow$ depends on the timeout for detecting lost "Hellos".\\
   		\textbf{Transient loops} appear due to inconsistent link-state database.
   		\includegraphics[width=\columnwidth]{images/Network_Layer/transient_loop.png} 
   		Convergence is the process during which routers seek to actively regain a consistent view of the network.\\
   		Network convergence time depends on 4 main factors:
   		\begin{itemize}[noitemsep]
   			\item \textbf{Detection}
   			\begin{itemize}
   				\item[$-$] realizing that a link/neighbor is down
   				\item[$-$] few ms
   				\item[$-$] smaller timers
   			\end{itemize} 
   			\item \textbf{Flooding}
   			\begin{itemize}
   				\item[$-$] flooding news to entire network
   				\item[$-$] few ms
   				\item[$-$] high-priority flooding
   			\end{itemize}  
   			\item \textbf{Computation}
   			\begin{itemize}
   				\item[$-$] recomputing Dijkstra
   				\item[$-$] few ms
   				\item[$-$] incremental algorithms
   			\end{itemize}
   			\item \textbf{Table update}
   			\begin{itemize}
   				\item[$-$] updating forward table
   				\item[$-$] potentially \textcolor{Red}{minutes}!
   				\item[$-$] better table design
   			\end{itemize}
   		\end{itemize} 
   		The problem with updating the forwarding table is that they are flat, means:
   		\vspace{-0.2cm}
   		\begin{itemize}[noitemsep]
   			\item[$-$] Entries do not share any information $\vert$ even if they are identical
   			\item[$-$] Upon failure, all of them have to be updated $\vert$ inefficient, but also unnecessary 
   		\end{itemize} 
   		The solution is to add a layer of \textbf{indirection} (Dereferenzierung):\\
   		Replay this:\\
   		\includegraphics[width=0.7\columnwidth]{images/Network_Layer/forward_table.png}\\
   		With that:\\
   		\includegraphics[width=\columnwidth]{images/Network_Layer/forward_table_pointer.png}
   		Now we only need to update 1 entry, the one in the pointer table! Hierarchical tables are able to converge within 150ms \textit{independently} on the number of prefixes!\\
   		Today, tow link-state protocols are widely used:
   		\begin{itemize}[noitemsep]
   			\item \textbf{OSPF}
   			\begin{itemize}
   				\item[$-$] used in many enterprise \& ISPs
   				\item[$-$] work on top of IP
   				\item[$-$] only route IPv4 by default
   			\end{itemize}
   			\item \textbf{IS-IS}
   			\begin{itemize}
   				\item[$-$] used mostly in large ISPs
   				\item[$-$] work on top of link layer
   				\item[$-$] network protocol agnostic
   			\end{itemize}
   		\end{itemize} 
		\par
		
		\textbf{Distance-vector protocols}\\
		Distance vector protocols are based on Bellman-Ford algorithm (see \ref{kap:How to compute forwarding} (3.))\\
		The same reasons as with the link-state cause the nodes to send new DVs.\\
		What happens when a link changes it cost?
		\vspace{-0.2cm}
		\begin{itemize}[noitemsep]
			\item Decrease cost
			\begin{itemize}
				\item[$-$] The algorithms terminates fast
				\item[$\rightarrow$] \textcolor{ForestGreen}{Good news travel fast}
			\end{itemize}
			\item Increase cost
			\begin{itemize}
				\item[$-$] The algorithm takes a while to terminate
				\item[$\rightarrow$] \textcolor{Red}{Bad news travel slow}
			\end{itemize}
		\end{itemize}
		This problem is known as \textcolor{Red}{count-to-infinity}, a type of routing loop.\\
		Solution: Whenever a router uses another one, it will announce it an infinite cost. This technique is known as \textbf{poisoned reverse}. This method does \textcolor{Red}{not} solve loops involving 3 or more nodes.\\
		Actual distance-vector protocols mitigate this issue by using small "infinity" (e.g "16").\\
		(Please see slides and exercises for examples of this problem and the solution method).\par
		
		\textbf{Link-state vs. Distance-vector routing:}
		\includegraphics[width=\columnwidth]{images/Network_Layer/state_vs_vector.png}
		 
		 \subsubsection{Inter-domain routing}
		 Find paths \textbf{between} networks.\\
		 The Internet is a network of neteorks, referred to as Autonomous Systems (AS). Each AS has a number which identifies it. \\
		 \textbf{BGP} is the routing protocol "gluing" the entire Internet together.
		 \includegraphics[width=\columnwidth]{images/Network_Layer/BGP_gluing.png}
		 Using BGP, ASes exchange information about the IP prefixes they can reach, directly or indirectly. \\
		 BGP needs to solve three key challenges:
		 \begin{itemize}[noitemsep]
		 	\item \textbf{Scalability}
		 	\begin{itemize}
		 		\item[$-$] There is a huge \# number of networks and prefixes
		 		\item[$-$] 700k prefixes, $>$50k networks, millions of routers 
		 	\end{itemize}
	 		\item \textbf{Privacy}
	 		\begin{itemize}
	 			\item[$-$] Networks do not want to divulge internal topologies
	 			\item[$-$] Or their business relationships
	 		\end{itemize}
 			\item \textbf{Policy enforcement}
 			\begin{itemize}
 				\item[$-$] Networks need to control where to send and receive traffic
 				\item[$-$] without an internet-wide notion of a link cost metric
 			\end{itemize}		
		 \end{itemize}
	 	Link-state and Distance-vector do both not solve these problems, but \textbf{BGP} (Border Gateway Protocol) does:
	 	%\vspace{-0.2cm}
	 	\begin{itemize}[noitemsep]
	 		\item BGP relies on \textbf{path-vector routing} to support flexible routing policies and avoid count-to-infinity.
	 		\item BGP announcements carry \textbf{complete path} information instead of distances.
	 		\item Each AS appends itself to the path when it propagates announcements
	 	\end{itemize}
	 	Complete path information enables ASes to easily detect a loop (if they see themselves in the path).\\
	 	Each AS can apply local routing policies, each AS is free to: 
	 	%\vspace{-0.2cm}
	 	\begin{itemize}[noitemsep]
	 		\item Select and use any path
	 		\begin{itemize}
	 			\item[$-$]prefarably the cheapest one
	 		\end{itemize}
 			\item Decide which path to export (if any) to which neighbor 
 			\begin{itemize}
 				\item preferably none to minimize carried traffic
 			\end{itemize}
	 	\end{itemize}
	 	\textbf{BGP}\\
	 	\vspace{-0.2cm}
	 	\begin{enumerate}[noitemsep]
	 		\item \textbf{BGP Policies} $\vert$ Follow the money
	 		\item \textbf{Protocol} $\vert$ How does it work
	 		\item \textbf{Problems} $\vert$ Security, performance 
	 	\end{enumerate}
	 	\textbf{1. Policies}\\
	 	BGP is a "follow the money" protocol. Two ASes connect only if they have a business relationship. There are two main business relationships today: 
	 	\begin{enumerate}[label=(\roman*),noitemsep]
	 		\item Customer/Provider
	 		\item Peer/Peer
	 	\end{enumerate}
	 	(i) Customer/Provider\\
	 	Customers pays providers to get full Internet connectivity (monthly bill).\\
	 	The amount paid is based on peak usage, usually according to the 95\% rule.\\
	 	Most ISPs discount unit price of data, if you pre-commit to a certain volume (Mengenrabatt).\par 
	 	
	 	(ii) Peer/Peer\\
	 	Peers \textbf{don't} pay each other for connectivity, they do it out of \textbf{common interest}. $\rightarrow$ If they exchange tons of traffic, they save money by connecting directly to each other.\par 
	 	
	 	Follow the money:\\ 
	 	\includegraphics[width=\columnwidth]{images/Network_Layer/follow_the_money.png}
	 	\textcolor{ForestGreen}{Allowed}: Swisscom/Salt/Sunrise make money, without paying someone to provide them. $\rightarrow$ Providers transit traffic for their customer.\\ 
	 	\textcolor{Red}{Forbidden 1}: Swisscom and Deutsche get money by sending over ETH, the ETH network will collapse, because all the traffic will be sent over them. $\rightarrow$ Customer do not transit traffic between their provider.\\
	 	\textcolor{Red}{Forbidden 2}: Sunrise won't receive upon, but has to provide content over their network. $\rightarrow$ Peers do not transit traffic between each other.\par 
	 	
	 	These policies are defined by constraining which BGP routes are \textbf{selected} and \textbf{exported}.\par 
	 	
	 	\textbf{Selection}\\
	 	Which path to use? \textcolor{Red}{Control outbound traffic}\\
	 	Selection rule For a destination p, prefer routes coming from: 
	 	\begin{itemize}[noitemsep]
	 		\item Customers over 
	 		\item Peers over 
	 		\item Providers
	 	\end{itemize}
	 	There will always be a provider route, but not necessarily a customer/peer route.\par 
	 	
	 	\textbf{Export}\\
	 	Which path to advertise? \textcolor{Red}{Control inbound traffic}\\
	 	Route exportation:\\
	 	\begin{tabular}{c | c c c c} 
	 			&  &  & send to &  \\ 
 				\hline
	 			&  & customer & peer & provider \\ 
	 			& costumer & \checkmark & \checkmark & \checkmark \\ 
	 			from & peer & \checkmark & - & - \\ 
	 			& provider & \checkmark & - & - \\ 
	 	\end{tabular} \\
 		\vspace{-0.2cm}
 		\begin{itemize}[noitemsep]
 			\item  Routes coming from customers are propagated to everyone else.
 			\item  Routes coming from peers and providers are only propagated to
 		\end{itemize}
 		\par
 		
 		\textbf{2. Protocol}\\
 		BGP sessions come in two flavors: 
 		\begin{itemize}[noitemsep]
 			\item externeal BGP (\textbf{eBGP})
 			\begin{itemize}
 				\item[$-$] Connect border routers in different ASes
 				\item[$-$] Used to learn routes to \textcolor{Red}{external destinations}\\
 				\includegraphics[width=0.8\columnwidth]{images/Network_Layer/eBGP_session.png}
 			\end{itemize}
 			\item internal BGP (\textbf{iBGP})
 			\begin{itemize}
 				\item[$-$] Connect the routers in the same AS
 				\item[$-$] Used to to disseminate (verbreiten) externally-learned routes internally
 				\item[$-$] Routes disseminated internally are then announced externally again using eBGP\\ 
 				\includegraphics[width=0.8\columnwidth]{images/Network_Layer/iBGP_session.png}
 			\end{itemize}
 		\end{itemize}        
		On the wire BGP is simple and only composed of four messages: 
		\begin{itemize}[noitemsep]
			\item Open 
			\begin{itemize}
				\item[$-$] establish TCP-based BGP sessions
			\end{itemize} 
			\item Notification 
			\begin{itemize}
				\item[$-$] report unusual conditions 
			\end{itemize}
			\item \textbf{Update} 
			\begin{itemize}
				\item[$-$] inform neighbor of a new best route
				\item[$-$] change in best route
				\item[$-$] removal of best route
			\end{itemize} 
			\item Keepalive
			\begin{itemize}
				\item[$-$] inform neighbor that connection is alive
			\end{itemize}
		\end{itemize}
		Update is the most important one. They carry an IP \textbf{prefix} together with a set of \textbf{attributes}. Attributes: 
		\begin{itemize}[noitemsep]
			\item \textbf{Next-Hop}
			\begin{itemize}
				\item[$-$] egress point identification
			\end{itemize}
			\item \textbf{AS-Path}
			\begin{itemize}
				\item[$-$] loop avoidance
				\item[$-$] outbound traffic control
				\item[$-$] inbound traffic control
			\end{itemize}
			\item \textbf{Local-Pref}
			\begin{itemize}
				\item[$-$] outbound traffic control
			\end{itemize}
			\item \textbf{Med}
			\begin{itemize}
				\item[$-$] inbound traffic control
			\end{itemize}
		\end{itemize}
		The \textbf{Next-Hop} is a global attribute which indicates where to send the traffic next. The Next-Hop is sent when the route enters an AS, it does not change within the AS. It only changes when going form one AS to an other!\\
		Solution:\\
		Force overwrite of Next-Hop to a virtual IP-address (\textbf{loopback} address) given to the router (unique identifier in internal network) and advertise this via OSPF in the internal network. Method is called \textcolor{Red}{\textbf{Next-Hop-Self}}.
		\includegraphics[width=\columnwidth]{images/Network_Layer/next_hop.png}
		\par 
		The \textbf{AS-Path} is a global attribute that lists all the ASes a route has traversed (in reverse order):\\
		\includegraphics[width=\columnwidth]{images/Network_Layer/as_path.png}
		\par 
		The \textbf{Local-Pref} is a \textit{local} attribute set at the border, it represents how \textit{preferred} a route is. Where do I want to send my traffic. Here all the routers will prefer DT over Swisscom:\\
		\includegraphics[width=\columnwidth]{images/Network_Layer/local_pref.png}
		\par 
		The \textbf{MED} is a global attribute which encodes the relative proximity of a prefix wrt to the announcer. You can \textbf{influence} your neighbor and try to tell them where to send your receiving traffic (Only works with multiple connections to the \textbf{same AS}). In the end though, the sender decides where he sends the traffic $\rightarrow$ I can influence not control:\\
		\begin{center}
			\includegraphics[width=0.7\columnwidth]{images/Network_Layer/med.png}
		\end{center}
		\par 
		Each BGP router processes updates according to a precise \textbf{pipeline}: 
		\includegraphics[width=\columnwidth]{images/Network_Layer/pipeline.png}
		\par 
		How to choose best route - prefer routes with:
		\vspace{-0.2cm} 
		\begin{enumerate}[noitemsep]
			\item higher Local-Pref
			\item shortest AS-Path length
			\item lower MED
			\item learned via eBGP instead of iGBP
			\item lower iGBP metric to next-hop
			\item smaller egress IP-address
		\end{enumerate}
		\par 
		\textbf{3. Problems}\\
		BGP suffers from many problems!
		\begin{enumerate}[label=(\roman*), noitemsep]
			\item \textbf{Reachability}
			\item \textbf{Security}
			\item \textbf{Convergence} 
			\item \textbf{Performance} 
			\item \textbf{Anomalies}
			\item \textbf{Relevance} 
		\end{enumerate}
	    i) \textbf{Reachability}\\
		Policy routing does not guarantee reachability, even if the network is physically connected.\par 
		
		ii) \textbf{Security}
		\vspace{-0.2cm}
		 \begin{itemize}[noitemsep]
		 	\item \textbf{IP Hijacking}
			\begin{itemize}
				\item[$-$] Steal IP-prefix and receive all the traffic that is not intended for you $\rightarrow$ Blackhole/Snooping/Impersonation 
			\end{itemize}
			\item \textbf{Bogus AS-paths}
			\begin{itemize}
				\item[$-$] Remove ASes form the path
				\item[$-$] Add Ases to the path
				\item[$-$] Add AS hop(s) at the end of the path
			\end{itemize}
			\item \textbf{Invalid paths}
			\begin{itemize}
				\item AS exports a route it shouldn't
			\end{itemize}
			\item \textbf{ Missing/Inconsistent routes}	
		\end{itemize}
		\par
		 
		iii) \textbf{Convergence}\\
		With arbitrary policies, BGP may have multiple stable states and never converges. It is possible that networks oscillate forever! Policy oscillations are a direct consequence of policy autonomy.
		\begin{itemize}[noitemsep]
			\item[$-$] Checking BGP correctness/convergence is not possible, even when you know all the policies!
		\end{itemize}
		\textbf{But}
		\begin{itemize}[noitemsep]
			\item[$-$] If all ASes follow the provider/peer/customer rules, BGP is \textbf{guaranteed} to converge, otherwise the network wouldn't be economical. 
		\end{itemize}
		\par 
		iv) \textbf{Performance}\\
		BGP path selection is mostly economical, not based on accurate performance criteria
		\includegraphics[width=\columnwidth]{images/Network_Layer/BGP_performance.png}
		\par 
		
		v) \textbf{Anomalies}\\
		BGP configuration is hard to get right. Human factors are responsible for 50\% to s80\% of network outages.
		\begin{itemize}[noitemsep]
			\item BGP is both bloated and underspecified
			\begin{itemize}
				\item[$-$] Lots of knobs and sometimes conflicting interpretations
			\end{itemize}
			\item BGP is often manually configured
			\begin{itemize}
				\item[$-$] Humans make often mistakes
			\end{itemize}
			\item BGP abstraction is fundamentally flawed
			\begin{itemize}
				\item[$-$] disjoint, router-based config. to effect AS-wide policy
			\end{itemize} 
		\end{itemize}
		\par 
		
		vi) \textbf{Relevance}\\
		The world of BGP policies is rapidly changing: 
		\begin{itemize}[noitemsep]
			\item ISPs are now eyeballs talking to content networks
			\begin{itemize}
				\item[$-$] e.g., Swisscom and Netflix/Youtube/Spotify
			\end{itemize}
			\item Transit becomes less important and less profitable
			\begin{itemize}
				\item[$-$] Traffic moves more and more to interconnection points
 			\end{itemize}
 			\item No system practices, yet 
 			\begin{itemize}
 				\item[$-$] details of peering arrangements are private anyway
 			\end{itemize}
		\end{itemize}
		
		\section{The Transport Layer} 
		\subsection{General stuff and context}
		What problems should be solved on transport layer? 
		\begin{itemize}[noitemsep]
			\item Data delivering to the correct application
			\begin{itemize}
				\item[$-$] IP just points towards next protocol
				\item[$-$] Transport need to demultiplex incoming data (ports)
			\end{itemize}
			\item Files or bytestreams abstractions for the applications
			\begin{itemize}
				\item[$-$] Network deals with packets
				\item[$-$] Transport layer need to translate between them
			\end{itemize}
			\item Reliable transfer (if needed)
			\item Not overloading the receiver
			\item Not overloading the network
		\end{itemize}
		What is needed to address these? 
		\begin{itemize}[noitemsep]
			\item Demultiplexing 
			\begin{itemize}
				\item[$-$] \textbf{Identifier for applicaiton process}
				\item[$-$] Going from host-to-host (IP) to process-to-process
			\end{itemize}
			\item Translating between bytestream and packets
			\begin{itemize}
				\item[$-$] \textbf{Do segmentation and reassembly}
			\end{itemize}
			\item Reliability
			\begin{itemize}
				\item[$-$] \textbf{ACKS and all that stuff}
			\end{itemize}
			\item Corruption
			\begin{itemize}
				\item[$-$] \textbf{Checksum}
			\end{itemize}
			\item Not overloading receivers
			\begin{itemize}
				\item[$-$] \textbf{Flow Control}
				\item[$-$] Limit data in receiver's buffer
			\end{itemize}
			\item Not overloading network
			\begin{itemize}
				\item[$-$] \textbf{Congestion Control}
			\end{itemize}
		\end{itemize}
		What transport protocols do \textbf{NOT} provide:
		\begin{itemize}[noitemsep]
			\item \textbf{Delay} and/or \textbf{bandwidth} guarantees
			\item Sessions that survive change of IP address
		\end{itemize}
		\textbf{Sockets}\\
		A socket is a software abstraction by which an application process exchanges network messages with the operating system. (\textbf{OS abstraction}).\\
		Two important types of sockets: 
		\begin{itemize}[noitemsep]
			\item UDP sockets
			\item TCP sockets
		\end{itemize}
		\par 
		
		\textbf{Ports}\\
		Is a \textbf{network abstraction}. Think of it as a \textbf{logical} interface on the host.\\
		Port is used as a transport layer identifier (16 bits) to tell which app (socket) gets which packets. OS stores mapping between sockets and ports.\\
		There exist some well known ports (0-1023):
		\begin{tabular}{l l}
			ssh & 22\\
			http & 80\\
			https & 443\\
			... & ...\\
		\end{tabular}\\
		Ports randomly given to clients lay in the range 1024-65535. 
		\subsection{UDP: User Datagram Protocol}
		UDP provides a \textbf{connectionless}/\textbf{unreliable} transport service.\\
		UDP provides only two services to the App. layer:
		\begin{itemize}[noitemsep]
			\item Multiplexing/Demultiplexing among processes
			\item Discarding corrupted packets (optional)
		\end{itemize}
		Therefore it is a very lightweight communication between processes.\\
		\textcolor{ForestGreen}{Advantages}: 
		\vspace{-0.2cm}
		\begin{itemize}[noitemsep]
			\item \textcolor{ForestGreen}{Finer Control over what data is sent and when}
			\begin{itemize}
				\item[$-$] As soon as an application process writes into a sockets UDP will package the data and send the packet
			\end{itemize}
			\item \textcolor{ForestGreen}{No delay for connection establishment}
			\begin{itemize}
				\item[$-$] UDP just blasts away without any formal preliminaries which avoids introducing any unnecessary delays. 
			\end{itemize}
			\item \textcolor{ForestGreen}{No connection state}
			\begin{itemize}
				\item[$-$] No allocation of buffers, sequence numbers, timers,... making it easier to handle many active clients at once
			\end{itemize}
			\item \textcolor{ForestGreen}{Small packet header overhead}
			\begin{itemize}
				\item[$-$] UDP header is only 8 bytes
			\end{itemize}
		\end{itemize}
		Popular applications that use UDP: 
		\begin{itemize}[noitemsep]
			\item VoIP
			\item Video conferencing
			\item online gaming
			\item streaming
			\item DNS
		\end{itemize}
	
		\subsection{TCP: Transmission Control Protocol}
		TCP provides a \textbf{connection-oriented, reliable, bytestream} transport service.
		\begin{itemize}[noitemsep]
			\item \textbf{Reliable, in-order delivery} (see also \ref{Subsec:Reliable_Delivery} )
			\begin{itemize}
				\item[$-$] Ensure byte stream arrives intact
				\begin{itemize}
					\item[$-$] ACKs
					\item[$-$] Checksums
					\item[$-$] Timeouts and retransmissions	
				\end{itemize}
			\end{itemize}
			\item \textbf{Connection oriented}
			\begin{itemize}
				\item[$-$] Set-up and tear-down of TCP session
			\end{itemize}
			\item \textbf{Full duplex stream of bytes service}
			\begin{itemize}
				\item[$-$] Sends ans receives stream of bytes, not messages
			\end{itemize}
			\item \textbf{Flow control} (see also \ref{Subsec:Reliable_Delivery})
			\begin{itemize}
				\item[$-$] Ensure taht sender doesn't overwhelm receiver
				\begin{itemize}
					\item[$-$] sliding window
					\item[$-$] Cumulative ACKs
				\end{itemize}
			\end{itemize}
			\item \textbf{Congestion Control}
			\begin{itemize}
				\item[$-$] Dynamic adaption to network path's capacity
			\end{itemize}
		\end{itemize}
		The TCP header looks as follows: \\
		\includegraphics[width=\columnwidth]{images/Transport_Layer/TCP_header.png}
		
		\subsubsection{Segments and Sequence Numbers}
		TCP stream-of-bytes service is provided by using segments:\\ 
		\includegraphics[width=\columnwidth]{images/Transport_Layer/segment_send.png}
		Segments are sent either if they reached their maximal size (\textbf{MSS} (usually 1460)) or a time out is reached. The segment is embedded in the IP packet:
		\includegraphics[width=\columnwidth]{images/Transport_Layer/TCP_in_IP.png}
		The MSS depends on the MTU and the IP/TCP-header:\\
		MSS = MTU $-$ (IP header) $-$ (TCP header)\\
		The \textbf{ISN} (Initial Sequence Number) identifies the first byte of the byte-stream. Because security reasons this number is picked randomly. \\
		The receiver will send an ACK with the sequence number of the \textbf{next expected} byte!:
		\includegraphics[width=\columnwidth]{images/Transport_Layer/ACK_seqnumb.png}
		
		\subsubsection{Connection Establishment/Teardown}
		To establish connection hosts exchange ISNs with the \textbf{SYN}:\\
		\begin{center}
			\includegraphics[height=0.5\columnwidth]{images/Transport_Layer/SYN.png}
		\end{center}
		If SYN gets lost, sender waits some time and transmits it again (hard to set timer reasonable).\\
		Tearing down connection can be made in three different ways: 
		\begin{itemize}[noitemsep]
			\item One side at a time (normal)
			\item Both together $\vert$ send FIN+ACK (normal)
			\item Abrupt $\vert$ RST (not normal)
		\end{itemize} 
		In the normal cases the sender of the last ACK stays some time in Zombi-Mode in case its ACK got lost. 
		\includegraphics[width=\columnwidth]{images/Transport_Layer/TCP_FIN.png}
		\par 
		
		\textbf{TCP state transitions}\\
		\includegraphics[width=\columnwidth]{images/Transport_Layer/state_transitions.png}
		
		\subsubsection{Timeouts and Retransmissions}	
		Setting the timeout value is very difficult, in order to do it measurements of the \textbf{RTT} (Round Trip Time) are used.
		To define the RTT an exponential averaging of the measurements is used:
		\includegraphics[width=\columnwidth]{images/Transport_Layer/RTT_calc.png} 
		Once a segment is retransmitted, it is not used for the measurement.\\
		A timeout is very expensive, therefore we rely mostly on duplicate ACKs:\\
		\textbf{Duplicate ACKs} are a sign of a isolated loss. One could trigger a resend upon recieving $k$ duplicate ACKs (TCP uses $k = 3$)
		
		\subsubsection{Congestion Control}
		Because of traffic burstiness and lack of BW reservation, congestion is inevitable and very harmful. Van Jacobsen saved us with Congestion Control.\\
		Congestion control aims at solving \textbf{three problems}:
		\begin{itemize}[noitemsep]
			\item \textbf{BW estimation}
			\begin{itemize}
				\item[$-$] How to adjust the BW of a single flow to the bottleneck BW?
			\end{itemize}
			\item \textbf{BW adaption}
			\begin{itemize}
				\item[$-$] How to adjust the BW of a single flow to variation of the bottleneck BW?
			\end{itemize}
			\item \textbf{Fairness}
			\begin{itemize}
				\item[$-$] How to share BW fairly among flows, without overloading the network.
			\end{itemize} 
		\end{itemize}
		Congestion control differs from flow control, TCP solves both using two distinct windows:
		\begin{itemize}[noitemsep]
			\item \textbf{Flow Control}
			\begin{itemize}
				\item[$-$] prevents one fast sender from overloading a slow receiver
				\item[$\rightarrow$] Solved using a receiving window (\textbf{RWND}) 
			\end{itemize}
			\item \textbf{Congestion Control} 
			\begin{itemize}
				\item[$-$] prevents a set of senders from overloading the network
				\item[$\rightarrow$] Solved using a congestion window (\textbf{CWND})
			\end{itemize}
		\end{itemize}
		Sender window = min(CWND, RWND)\par 
		
		\textbf{Detecting congestion}\\
		There are essentially three ways to detect congestion:
		\begin{itemize}[noitemsep]
			\item[$-$] Network could tell the source
			\begin{itemize}
				\item[$-$] \textcolor{Red}{Signal could be lost}
			\end{itemize}
			\item[$-$] Measure packet delay
			\begin{itemize}
				\item[$-$] \textcolor{Red}{Signal is noisy}
			\end{itemize}
			\item[$-$] Measure packet loss
			\begin{itemize}
				\item[$-$] \textcolor{ForestGreen}{Fail-safe signal that TCP already has to detect}
			\end{itemize}
		\end{itemize}
		$\rightarrow$ Packet dropping is the best solution! Detecting losses can be done using \textbf{ACKs} or \textbf{timeouts}, the two signal differ in their degree of severity:
		\begin{itemize}[noitemsep]
			\item Duplicated ACKs
			\begin{itemize}
				\item[$-$] \textbf{mild} congestion signal
				\item[$\rightarrow$] packets are still making it
			\end{itemize}
			\item Timeout
			\begin{itemize}
				\item[$-$] \textbf{Severe} congestion signal
				\item[$\rightarrow$] multiple consequent losses
			\end{itemize}
		\end{itemize} 
		\par
		
		\textbf{Reacting to congestion}\\
		TCP approach is to \textbf{gently increase} when not congested and to \textbf{rapidly decrease} when congested.\par 
		
		1) Get a first order estimate of the available BW.\\
		$\rightarrow$ start slow but rapidly increase until a packet drop occurs. This increase phase, known as \textbf{slow start}, corresponds to an exponential increase of the CWND. The problem with slow start is that it can result in full window of packet loss $\rightarrow$ we need a more gentle adjustment once we have a rough estimate of the BW.\\ 
		\includegraphics[width=\columnwidth]{images/Transport_Layer/slow_start.png}
		\par 
		
		2)Track the available BW and oscillate around its current value\\
		$\rightarrow$ Tow possible variations: 
		\begin{itemize}[noitemsep]
			\item[$-$] Multiple Decrease/Increase (MD/MI)
			\item[$-$] Additive Decrease/Increase (AD/AI)
		\end{itemize} 
		This leads to four alternative designs: AIAD, AIMD, MIAD, MIMD. To select one scheme, we need to consider \textbf{fairness} $\rightarrow$ two identical flows should end up with the same BW. In the following graph we can plot the system trajectories and analyze the systems behavior. The blue dots are the result of the AIMD scheme.\\
		\includegraphics[width=\columnwidth]{images/Transport_Layer/trajectory_plot.png}
		In practice TCP implements \textbf{AIMD} (gentle increase / aggressive decrease), because it converges to fairness and efficiency and then oscillates around the optimum in a stable way.\\
		TCP congestion control almost complete:
		\vspace{0.1cm}
		\begin{center}
			\includegraphics[width=0.9\columnwidth]{images/Transport_Layer/congestion_code.png}
		\end{center}
		\vspace{0.1cm}
		Congestion control makes TCP throughput look like a sawtooth:\\
		\includegraphics[width=\columnwidth]{images/Transport_Layer/tcp_throughput.png}
		\par 
		
		\section{The Application Layer}
		\subsection{DNS}
		Internet has one global system for naming hosts $\rightarrow$ DNS. The DNS system is a distributed database which enables to resolve a name into an IP address.\\
		\includegraphics[width=\columnwidth]{images/Application_Layer/dns_ip.png}
		In practice, names can be mapped to more than one IP and IPs can be mapped by more than one name.\\
		To scale, DNS adopt three intertwined hierarchies: 
		\begin{itemize}[noitemsep]
			\item Naming structure
			\begin{itemize}
				\item[$-$] Hierarchy of addresses
			\end{itemize}
			\item Management
			\begin{itemize}
				\item[$-$] Hierarchy of authority over names 
			\end{itemize}
			\item Infrastructure
			\begin{itemize}
				\item[$-$] Hierarchy of DNS servers
			\end{itemize}
		\end{itemize}
		The following tree structure shows the hierarchy of addresses, management and infrastructure.   
		\includegraphics[width=\columnwidth]{images/Application_Layer/dns_tree.png}
		There are 13 root-servers (managed professionally).\\
		Top Level Domain (TLD) servers are also managed professionally by private or non-profit organizations.\\
		The bottom (and bulk) of the hierarchy is managed by Internet Service Provider or locally. \\
		Every server knows the address of the root servers.\\
		Each root server knows the address of all TLD servers. From there on, each server knows the address of all children.\\
		DNS query and reply uses UDP port 53, reliability is implemented by repeating requests.\\
		A DNS server stores Resource Records composed of a name, value, type, TTL.
		\vspace{0.2cm}
		
		\begin{tabular}{l l l}
			\textbf{Records} & \textbf{Name}  & \textbf{Value} \\ 
			A	  & hostname   & IP address  \\ 
			NS	  & domain 	   & DNS sevrer name \\ 
			MX	  & domain 	   & Mail server name \\ 
			CNAME & alias 	   & Canonical name \\  
			PTR	  & IP address & corresponding hostname \\ 
		\end{tabular} 
	
		DNS resolution can either be \textbf{recursive} or \textbf{iterative}:\par
		
		\textbf{Recursive}\\
		When performing a recursive query, the client offloads  the task of resolving to the server. 
		Because of security and scalability reasons, this is almost never used.\\
		\includegraphics[width=\columnwidth]{images/Application_Layer/dns_recursive.png}
		\par 
		
		\textbf{Iterative}\\
		When performing an iterative query, the server only returns the address of the next server to query.\\
		\includegraphics[width=\columnwidth]{images/Application_Layer/dns_iterative.png}
		
		
		
		\end{multicols*}
	\setcounter{secnumdepth}{3}
\end{document}
